\documentclass{report}

\usepackage{amsfonts, amsmath, amssymb, amsthm}
\usepackage[margin=1.0in]{geometry}
\usepackage{hyperref}
\usepackage{float}
\usepackage{fancyhdr}
\usepackage{graphicx}
\usepackage{mathrsfs}

\newcommand{\M}[2]{\mathbb{#1}^{#2}}
\newcommand{\twovector}[2]{\left[ \begin{array}{c} #1 \\ #2 \\ \end{array} \right]}
\newcommand{\threevector}[3]{\left[ \begin{array}{c} #1 \\ #2 \\ #3 \\ \end{array} \right]}
\newcommand{\fourvector}[4]{\left[ \begin{array}{c} #1 \\ #2 \\ #3 \\ #4 \\ \end{array} \right]}
\newcommand{\fivevector}[5]{\left[ \begin{array}{c} #1 \\ #2 \\ #3 \\ #4 \\ #5 \\ \end{array} \right]}
\newcommand{\sixvector}[6]{\left[ \begin{array}{c} #1 \\ #2 \\ #3 \\ #4 \\ #5 \\ #6 \\ \end{array} \right]}
\newcommand{\sevenvector}[7]{\left[ \begin{array}{c} #1 \\ #2 \\ #3 \\ #4 \\ #5 \\ #6 \\ #7 \\ \end{array} \right]}
\newcommand{\eightvector}[8]{\left[ \begin{array}{c} #1 \\ #2 \\ #3 \\ #4 \\ #5 \\ #6 \\ #7 \\ #8 \\ \end{array} \right]}
\newcommand{\ninevector}[9]{\left[ \begin{array}{c} #1 \\ #2 \\ #3 \\ #4 \\ #5 \\ #6 \\ #7 \\ #8 \\ #9 \\ \end{array} \right]}
\newcommand{\nbrack}[1]{\left( #1 \right)}
\newcommand{\bbrack}[1]{\left[ #1 \right]}
\newcommand{\cbrack}[1]{\left\lbrace #1 \right\rbrace}
\newcommand{\abrack}[1]{\left< #1 \right>}
\newcommand{\linebrack}[1]{\left| #1 \right|}
\newcommand{\twomatrix}[4]{\bbrack{
    \begin{array}{cc}
      #1 & #2 \\
      #3 & #4 \\
    \end{array}
  }
}
\newcommand{\threematrix}[9]{\bbrack{
    \begin{array}{ccc}
      #1 & #2 & #3 \\
      #4 & #5 & #6 \\
      #7 & #8 & #9 \\
    \end{array}
  }
}
\newcommand{\Lplc}[1]{\mathscr{L}\bbrack{ #1 } (s)}
\newcommand{\iLplc}[1]{\mathscr{L}^{-1}\bbrack{ #1 } (t)}
\newcommand{\Var}[1]{\text{Var} \bbrack{ #1 }}
\newcommand{\std}[1]{\text{E} \bbrack{ #1 }}
\newcommand{\Prob}[1]{\text{P} \bbrack{ #1 }}

\title{Innlevering 5}
\author{Jacob Oliver Bruun}
\date{\today}

\makeatletter
\let\inserttitle\@title
\let\insertauthor\@author
\makeatother

\pagestyle{fancy}
\chead{\insertauthor}
\lhead{\inserttitle}
\rhead{\today}

\parindent 0ex

\begin{document}

\section*{Oppg. 1}
Har en normalfordeling med ukjent $\mu$ og $\sigma^{2} = 0.060^{2}$ \\

\textbf{a)}
Antar $\mu = 6.8$ og har da
\begin{equation}
  \label{eq:1}
  \Prob{ X < 6.74 } = \Prob{ \frac{X-\mu}{\sigma} < \frac{6.74 - \mu}{\sigma}} = \Prob{ Z < -1 } = 0.1587
\end{equation}
ved å slå opp i tabell. Kan så finne $\Prob{6.74 < X < 6.86}$, men siden normalfordelingen er lik på begge sider av $\mu$ har vi
\begin{equation}
  \label{eq:2}
  \Prob{6.74 < X < 6.86} = 1 - 2\Prob{ X < 6.74 } = 0.6826
\end{equation}
til slutt ser vi på $\Prob{ |X - \mu| > 0.06 }$ og har da
\begin{equation}
  \label{eq:3}
  \Prob{ |X - \mu| > 0.06 } = \Prob{X < 6.74} + \Prob{X > 6.86} = 1 - \Prob{6.74 < X < 6.86} = 0.3174
\end{equation}

\textbf{b)}
Ser på $Y = \sum_{i=1}^{5} X_{i}/5$ og vet ikke lenger hva $\mu$ er, kan dermed se på
\begin{equation}
  \label{eq:5}
  \Var{Y} = \frac{1}{25} \bbrack{ 5\sigma^{2} } = \frac{1}{5} \sigma^{2}
\end{equation}
vet at $Y$ er normalfordelt og at det ikke har noe å si hvor $\mu$ er og kan dermed anta $\mu = 0$ og da
\begin{equation}
  \label{eq:20}
  \Prob{|Y| > 0.06} = 2\Prob{Y > 0.06} = 2\Prob{\frac{\sqrt{5} Y}{\sigma} > \frac{0.06\sqrt{5}}{\sigma}} = 0.025
\end{equation}
ved oppslag i tabell og ser videre på
\begin{equation}
  \label{eq:18}
  \begin{split}
    \overline{x} - z_{0.025}\frac{\sigma}{\sqrt{n}} &= 6.707 \\
    \overline{x} + z_{0.025}\frac{\sigma}{\sqrt{n}} &= 6.813 \\
  \end{split}
\end{equation}
og har dermed konfidensintervallet på $95\%$
\begin{equation}
  \label{eq:19}
  \mu \in \nbrack{ 6.707, 6.813 }
\end{equation}



\section*{Oppg. 2}
\textbf{a)}
Ser på poissonfordelingen
\begin{equation}
  \label{eq:6}
  f(x) = \frac{\lambda^{x}}{x!} e^{-\lambda}, x\in \M{N}{}
\end{equation}
antar $\lambda = 15$ og ser da at vi har
\begin{equation}
  \label{eq:7}
  \std{X} = \mu = \lambda = 15
\end{equation}
og har dermed
\begin{equation}
  \label{eq:8}
  \Prob{X>20} = 1 - \Prob{X\leq 20} = 0.083
\end{equation}
og
\begin{equation}
  \label{eq:9}
  \Prob{10\leq X < 20} = \Prob{X<20} - \Prob{X < 10} = 0.8053
\end{equation}

\textbf{b)}
Antar så at vi ikke kjenner $\lambda$ og har $\hat{\lambda} = \overline{X} = 359/30$ og har da fra sentralgrenseteoremet at dette kan tilnærmes en standard normalfordeling og kan se på
\begin{equation}
  \label{eq:10}
  Z = \frac{\overline{X} - \lambda}{\sigma}\sqrt{n}
\end{equation}
og vi bruker variansen i en poissonfordeling som $\sigma$ og dermed
\begin{equation}
  \label{eq:13}
  \sigma^{2} = \Var{X} = \overline{X}
\end{equation}
vil ha et konfidensintervall med $99\%$ sannsynlighet og må dermed ha
\begin{equation}
  \label{eq:11}
  \begin{split}
    \Prob{ -t_{0.005} < T < t_{0.005} } &= 0.99 \\
    \Prob{ \overline{X} - t_{0.005} \frac{\sigma}{\sqrt{n}} < \lambda < \overline{X} + t_{0.005} \frac{\sigma}{\sqrt{n}} } &= 0.99
  \end{split}
\end{equation}
slår opp i tabell og finner $t_{0.005} = 2.750$ med frihetsgrad $\nu = 30$ og har dermed
\begin{equation}
  \label{eq:12}
  \lambda \in \nbrack{10.34, 13.59}
\end{equation}

\textbf{c)}
Ser så på sannsynlighetsmaksimeringsestimatoren for $\lambda$ og definerer da
\begin{equation}
  \label{eq:24}
  \begin{split}
    L(x_{1}, x_{2}, \dots , x_{n} , y_{1}, y_{2}, \dots, y_{m}; \lambda)
    &= \prod_{i = 1}^{n} f(x_{i}; \lambda) \cdot \prod_{j=1}^{m} f(y_{j}; \lambda) \\
    &= \lambda^{\sum_{i=1}^{n} x_{i}} \frac{e^{-n\lambda}}{\prod_{i=1}^{n}x_{i}!} \cdot \nbrack{\frac{\lambda}{2}}^{\sum_{j=1}^{m}y_{j}} \frac{e^{-\frac{m\lambda}{2}}}{\prod_{j=1}^{m}y_{j}!} \\
  \end{split}
\end{equation}
og finner så logaritmen av $L$
\begin{equation}
  \label{eq:25}
  \ln L = \sum_{i=1}^{n} x_{i} \ln \lambda - n\lambda - \ln \prod_{i=1}^{n} x_{i}! + \sum_{j=1}^{m} y_{j} \ln \frac{\lambda}{2} - \frac{m\lambda}{2} - \ln \prod_{j=1}^{m} y_{j}!
\end{equation}
og deriverer så med hensyn på $\lambda$
\begin{equation}
  \label{eq:26}
  \begin{split}
    \frac{\partial L}{\partial \lambda} = 0 &= \sum_{i=1}^{n} x_{i} \frac{1}{\lambda} - n + \sum_{j=1}^{m} y_{j} \frac{1}{\lambda} - \frac{m}{2} \\
    \lambda \nbrack{ n + \frac{1}{2}m } &= \sum_{i=1}^{n} x_{i} + \sum_{j=1}^{m} y_{j} \\
    \lambda &= \frac{\sum_{i=1}^{n} x_{i} + \sum_{j=1}^{m} y_{j}}{n + \frac{1}{2}m} \\
    \lambda &= \frac{\overline{x}n + \overline{y}m}{n + \frac{1}{2}m}
  \end{split}
\end{equation}
og har dermed
\begin{equation}
  \label{eq:27}
  \hat{\lambda} = \frac{\overline{x}n + \overline{y}m}{n + \frac{1}{2}m}
\end{equation}



\section*{Oppg. 3}
\textbf{a)}
Ser at $X$ er definert slik at for hver del av DNA-strukturen så er det to muligheter, at det er en match eller ikke. Siden det sjekkes for $n$ deler som er uavhengige er $X$ binomisk fordelt
\begin{equation}
  \label{eq:4}
  f(x) = b(x; n, p) = \binom{n}{x} p^{x}(1-p)^{n-x}, x \in \M{N}{} \cup \cbrack{0}
\end{equation}
kan dermed finne ved $n=5$
\begin{equation}
  \label{eq:14}
  \Prob{X=2} = f(2) = \frac{5!}{2!(5-2)!} \cdot 0.15^{2} \cdot (1-0.15)^{5-2} = 0.138
\end{equation}
og
\begin{equation}
  \label{eq:15}
  \Prob{X \geq 2} = 1 - \Prob{X=1} - \Prob{X=0} = 1 - 5\cdot 0.15 \cdot 0.85^{4} - 0.85^{5} = 0.165
\end{equation}
og
\begin{equation}
  \label{eq:16}
  \Prob{X = 2 | X \geq 2} = \frac{\Prob{X=2}}{\Prob{X \geq 2}} = 0.836
\end{equation}

\textbf{b)}
Ser på sjansen for en type 1 feil og har da



\section*{Oppg. 4}
Har $X$ med normalfordeling med $\mu = 80, \sigma^{2} = 18^{2}$ og $Y = \sum_{n=1}^{16} X_{n}$ ser på
\begin{equation}
  \label{eq:17}
  \Prob{X_{1} > 90} = \Prob{\frac{X_{1} - \mu}{\sigma} > \frac{90-\mu}{\sigma}} = 0.2877
\end{equation}
og vet at
\begin{equation}
  \label{eq:21}
  \mu_{Y} = 16 \mu_{X}, \;\;\;\; \sigma_{Y}^{2} = 16\sigma_{X}^{2}
\end{equation}
har da
\begin{equation}
  \label{eq:22}
  \std{Y} = 16\mu_{X} = 1280
\end{equation}
og
\begin{equation}
  \label{eq:23}
  \Var{Y} = 16\sigma_{X}^{2} = 4^{2} \cdot 18^{2} = 72^{2}
\end{equation}
og til slutt fordi $Y$ også er normalfordelt
\begin{equation}
  \label{eq:28}
  \Prob{Y > 16\cdot 90} = \Prob{\frac{Y-\mu_{Y}}{\sigma_{Y}}\sqrt{n} > \frac{16\cdot 90 - \mu_{Y}}{\sigma_{Y}}\sqrt{n}} = \Prob{Z > 2.22} = 0.013
\end{equation}
ved oppslag i tabell. Dersom vi hadde sett bortifra uavhengighet kunne vi ikke brukt regnereglene for varians og forventningsverdi og de tre siste utregningene ville ikke vært rett. Vi trenger så vite sannsynlighetsfordelingen for å finne sannsynlighetene $\Prob{X_{1} > 90}, \Prob{Y > 16\cdot 90}$, men variansen og forventningsverdien er ikke avhengig av dette og disse vil da fortsatt stemme.

\textbf{b)}
Ser at vi kan skrive
\begin{equation}
  \label{eq:29}
  \overline{x} = \frac{1591}{20}, S^{2} = \frac{1}{19} \sum_{i=1}^{20} \nbrack{ x_{i} - \overline{x} }^{2} = 192.471
\end{equation}
finner så et $90\%$ konfidensintervall
\begin{equation}
  \label{eq:30}
  \begin{split}
    \overline{x} - t_{0.05} \frac{S}{\sqrt{n}} &= 74.20 \\
    \overline{x} + t_{0.05} \frac{S}{\sqrt{n}} &= 84.90 \\
  \end{split}
\end{equation}
og har dermed
\begin{equation}
  \label{eq:31}
  \mu \in \nbrack{74.20, 84.90}
\end{equation}



\end{document}
