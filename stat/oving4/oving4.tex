\documentclass{report}

\usepackage{amsfonts, amsmath, amssymb, amsthm}
\usepackage[margin=1.0in]{geometry}
\usepackage{hyperref}
\usepackage{float}
\usepackage{fancyhdr}
\usepackage{graphicx}
\usepackage{mathrsfs}
\usepackage{comment}

\newcommand{\M}[2]{\mathbb{#1}^{#2}}
\newcommand{\twovector}[2]{\left[ \begin{array}{c} #1 \\ #2 \\ \end{array} \right]}
\newcommand{\threevector}[3]{\left[ \begin{array}{c} #1 \\ #2 \\ #3 \\ \end{array} \right]}
\newcommand{\fourvector}[4]{\left[ \begin{array}{c} #1 \\ #2 \\ #3 \\ #4 \\ \end{array} \right]}
\newcommand{\fivevector}[5]{\left[ \begin{array}{c} #1 \\ #2 \\ #3 \\ #4 \\ #5 \\ \end{array} \right]}
\newcommand{\sixvector}[6]{\left[ \begin{array}{c} #1 \\ #2 \\ #3 \\ #4 \\ #5 \\ #6 \\ \end{array} \right]}
\newcommand{\sevenvector}[7]{\left[ \begin{array}{c} #1 \\ #2 \\ #3 \\ #4 \\ #5 \\ #6 \\ #7 \\ \end{array} \right]}
\newcommand{\eightvector}[8]{\left[ \begin{array}{c} #1 \\ #2 \\ #3 \\ #4 \\ #5 \\ #6 \\ #7 \\ #8 \\ \end{array} \right]}
\newcommand{\ninevector}[9]{\left[ \begin{array}{c} #1 \\ #2 \\ #3 \\ #4 \\ #5 \\ #6 \\ #7 \\ #8 \\ #9 \\ \end{array} \right]}
\newcommand{\nbrack}[1]{\left( #1 \right)}
\newcommand{\bbrack}[1]{\left[ #1 \right]}
\newcommand{\cbrack}[1]{\left\lbrace #1 \right\rbrace}
\newcommand{\abrack}[1]{\left< #1 \right>}
\newcommand{\linebrack}[1]{\left| #1 \right|}
\newcommand{\twomatrix}[4]{\bbrack{
    \begin{array}{cc}
      #1 & #2 \\
      #3 & #4 \\
    \end{array}
  }
}
\newcommand{\threematrix}[9]{\bbrack{
    \begin{array}{ccc}
      #1 & #2 & #3 \\
      #4 & #5 & #6 \\
      #7 & #8 & #9 \\
    \end{array}
  }
}
\newcommand{\Lplc}[1]{\mathscr{L}\bbrack{ #1 } (s)}
\newcommand{\iLplc}[1]{\mathscr{L}^{-1}\bbrack{ #1 } (t)}
\newcommand{\Var}[1]{\text{Var} \bbrack{ #1 }}
\newcommand{\std}[1]{\text{E} \bbrack{ #1 }}
\newcommand{\Prob}[1]{\text{P} \bbrack{ #1 }}

\title{Innlevering 4}
\author{Jacob Oliver Bruun}
\date{\today}

\makeatletter
\let\inserttitle\@title
\let\insertauthor\@author
\makeatother

\pagestyle{fancy}
\chead{\insertauthor}
\lhead{\inserttitle}
\rhead{\today}

\parindent 0ex

\begin{document}

\section*{Oppg. 1}
\textbf{a)}
Har en normalfordeling med $\mu = 10, \sigma^{2} = 0.2^{2}$, ser først på
\begin{equation}
  \label{eq:1}
  P(X > 10.2) = \frac{1}{\sqrt{2\pi}\sigma} \int_{10.2}^{\infty} e^{-\frac{\nbrack{ x - \mu }^{2}}{2\sigma^{2}}} dx \approx 0.159
\end{equation}
ved å sette in i WolframAlpha. Siden dette er en symmetrisk sansynlighetsfordeling har vi
\begin{equation}
  \label{eq:2}
  P\nbrack{ |X - \mu| > 0.2} = 2\cdot P(X > 10.2) = 0.318.
\end{equation}
Kan sette et forsøk $Y = 2X$ og har da
\begin{equation}
  \label{eq:3}
  F_{Y}(y) = \int_{-\infty}^{\frac{y}{2}} \frac{1}{\sqrt{2\pi}\sigma} e^{-\frac{\nbrack{ x - \mu }^{2}}{2\sigma^{2}}} dx
\end{equation}
som gir
\begin{equation}
  \label{eq:4}
  P(|Y-\mu | > 0.2) = 2 P(Y < 19.6) = F_{Y}(19.6) \approx 0.159
\end{equation}
også ved å bruke WolframAlpha. \\

\textbf{b)}
Ser på forsøk 1 og vet at dette ikke er forskjellig fra det opprinnelige forsøket og har da
\begin{equation}
  \label{eq:5}
  \std{ \hat{\mu}_{A}} = \mu_{A}, \;\; \std{\hat{\mu}_{B}} = \mu_{B}, \;\; \Var{\hat{\mu}_{A}} = \sigma^{2}, \;\; \Var{\hat{\mu}_{B}} = \sigma^{2}.
\end{equation}
Ser så på forsøk 2 og setter $Y_{1} = X_{1} + X_{2}$ og $Y_{2} = X_{1} - X_{2}$



\end{document}
