\documentclass{report}

\usepackage{amsfonts, amsmath, amssymb, amsthm}
\usepackage[margin=1.0in]{geometry}
\usepackage{hyperref}
\usepackage{float}
\usepackage{fancyhdr}
\usepackage{graphicx}
\usepackage{mathrsfs}

\newcommand{\M}[2]{\mathbb{#1}^{#2}}
\newcommand{\twovector}[2]{\left[ \begin{array}{c} #1 \\ #2 \\ \end{array} \right]}
\newcommand{\threevector}[3]{\left[ \begin{array}{c} #1 \\ #2 \\ #3 \\ \end{array} \right]}
\newcommand{\fourvector}[4]{\left[ \begin{array}{c} #1 \\ #2 \\ #3 \\ #4 \\ \end{array} \right]}
\newcommand{\fivevector}[5]{\left[ \begin{array}{c} #1 \\ #2 \\ #3 \\ #4 \\ #5 \\ \end{array} \right]}
\newcommand{\sixvector}[6]{\left[ \begin{array}{c} #1 \\ #2 \\ #3 \\ #4 \\ #5 \\ #6 \\ \end{array} \right]}
\newcommand{\sevenvector}[7]{\left[ \begin{array}{c} #1 \\ #2 \\ #3 \\ #4 \\ #5 \\ #6 \\ #7 \\ \end{array} \right]}
\newcommand{\eightvector}[8]{\left[ \begin{array}{c} #1 \\ #2 \\ #3 \\ #4 \\ #5 \\ #6 \\ #7 \\ #8 \\ \end{array} \right]}
\newcommand{\ninevector}[9]{\left[ \begin{array}{c} #1 \\ #2 \\ #3 \\ #4 \\ #5 \\ #6 \\ #7 \\ #8 \\ #9 \\ \end{array} \right]}
\newcommand{\nbrack}[1]{\left( #1 \right)}
\newcommand{\bbrack}[1]{\left[ #1 \right]}
\newcommand{\cbrack}[1]{\left\lbrace #1 \right\rbrace}
\newcommand{\abrack}[1]{\left< #1 \right>}
\newcommand{\linebrack}[1]{\left| #1 \right|}
\newcommand{\twomatrix}[4]{\bbrack{
    \begin{array}{cc}
      #1 & #2 \\
      #3 & #4 \\
    \end{array}
  }
}
\newcommand{\threematrix}[9]{\bbrack{
    \begin{array}{ccc}
      #1 & #2 & #3 \\
      #4 & #5 & #6 \\
      #7 & #8 & #9 \\
    \end{array}
  }
}
\newcommand{\Lplc}[1]{\mathscr{L}\bbrack{ #1 } (s)}
\newcommand{\iLplc}[1]{\mathscr{L}^{-1}\bbrack{ #1 } (t)}
\newcommand{\Var}[1]{\text{Var} \bbrack{ #1 }}
\newcommand{\fvv}[1]{\text{E} \bbrack{ #1 }}
\newcommand{\Prob}[1]{\text{P} \bbrack{ #1 }}

\title{Innlevering 5}
\author{Jacob Oliver Bruun}
\date{\today}

\makeatletter
\let\inserttitle\@title
\let\insertauthor\@author
\makeatother

\pagestyle{fancy}
\chead{\insertauthor}
\lhead{\inserttitle}
\rhead{\today}

\parindent 0ex

\begin{document}
\section*{Oppg. 1}
\textbf{a)}
Har en Poisson-fordeling og definerer da
\begin{equation}
  \label{eq:1}
  L(\mathbf{x}, \lambda_{R}) = f(x_{1}, \lambda_{R}) \cdots f(x_{n}, \lambda_{R})
\end{equation}
med
\begin{equation}
  \label{eq:2}
  f(x_{i}, \lambda_{R}) = \frac{\lambda_{R}^{x_{i}}}{x_{i}!} e^{-\lambda_{R}}
\end{equation}
og finner så
\begin{equation}
  \label{eq:3}
  \frac{\partial \ln L}{\partial \lambda_{R}} = \frac{\partial}{\partial \lambda_{R}} \cbrack{ \ln \bbrack{ \frac{\lambda_{R}^{\sum_{i=1}^{n} x_{i}}}{\prod_{j=1}^{n}x_{j}!} e^{-n\lambda_{R}} } } = \frac{\partial}{\partial \lambda_{R}} \cbrack{ \ln \lambda_{R} \sum_{i=1}^{n}x_{i} - \ln \prod_{j=1}^{n}x_{j}! - n\lambda_{R}} = \frac{1}{\lambda_{R}} \sum_{i=1}^{n}x_{i} - n
\end{equation}
setter vi så likning \eqref{eq:3} lik 0 finner vi
\begin{equation}
  \label{eq:4}
  \hat{\lambda}_{R} = \frac{1}{n} \sum_{i=1}^{n} x_{i}
\end{equation}
ser også at $\hat{\lambda}_{R} = \overline{X}$ og har da
\begin{equation}
  \label{eq:5}
  \fvv{\hat{\lambda}_{R}} = \fvv{ \frac{1}{n} \sum_{i=1}^{n} x_{i} } = \frac{1}{n} \fvv{ \sum_{i=1}^{n} x_{i}} = \frac{1}{n} \cdot n\lambda_{R} = \lambda_{R}
\end{equation}
og ser så på
\begin{equation}
  \label{eq:6}
  \Var{\hat{\lambda}_{R}} = \Var{\frac{1}{n} \sum_{i=1}^{n} x_{i}} = \frac{1}{n^{2}} \Var{\sum_{i=1}^{n} x_{i}} = \frac{1}{n^{2}} \cdot n\lambda_{R} = \frac{\lambda_{R}}{n}
\end{equation}
ved sentralgrenseteoremet er dette normalfordelt. \\

\textbf{b)}
Setter da opp hypotesene
\begin{equation}
  \label{eq:7}
  H_{0} : \lambda_{R} = 10, \;\;\;\; H_{1} : \lambda_{R} < 10
\end{equation}
og kan sette opp ved sentralgrenseteoremet
\begin{equation}
  \label{eq:8}
  T = \frac{\overline{X} - \lambda_{R}}{\sqrt{\Var{X}}} \approx -1.34164
\end{equation}
for $H_{0}$ og kan så finne $p$-verdien
\begin{equation}
  \label{eq:9}
  p = \Prob{ T \leq -1.34164 } = 0.0901
\end{equation}
og har dermed med $\alpha = 0.05$ at vi ikke forkaster hypotesen. \\

\textbf{c)}
Antar $\lambda_{R} = 9$ og fortsetter å operere med $\alpha = 0.05$ og vil finne
\begin{equation}
  \label{eq:13}
  F(x) = \sum_{n=0}^{x} f(x) = e^{-9} \sum_{n=0}^{x} \frac{9^{n}}{n!} = 0.9
\end{equation}
og finner ved tabell $x = 13$
\begin{equation}
  \label{eq:11}
  p = \Prob{ T \leq -\frac{1}{3}\sqrt{n} } \leq 0.05
\end{equation}
og ved å se i tabell trenger vi da
\begin{equation}
  \label{eq:12}
  -\frac{1}{3}\sqrt{n} = -1.65 \Rightarrow
\end{equation}



\section*{Oppg. 2}
Ser på $Y_{i} = ax_{i}(1-x_{i}) + \epsilon_{i}$ for $i = 1, 2, \dots, n$ og har
\begin{equation}
  \label{eq:14}
  \fvv{ Y_{i} | x_{i} } = ax_{i} (1-x_{i}), \;\; \Var{ Y_{i} | x_{i} } = \sigma_{0}^{2} = 0.025^{2}
\end{equation}
ser på sannsynlighetsmaksimeringsestimatoren for $a$ og definerer
\begin{equation}
  \label{eq:15}
  L(\mathbf{x}, a) = \prod_{i=1}^{n} \bbrack{ ax_{i} (1-x_{i}) + \epsilon_{i} }
\end{equation}
og setter
\begin{equation}
  \label{eq:16}
  \begin{split}
    0 = \frac{\partial \ln L}{\partial a}
    &= \frac{\partial}{\partial a} \ln \cbrack{ \prod_{i=1}^{n} \bbrack{ ax_{i} (1-x_{i}) + \epsilon_{i} } } \\
    &= \frac{\partial}{\partial a} \sum_{i=1}^{n}\ln \bbrack{ ax_{i}(1-x_{i}) + \epsilon_{i} } \\
    &= \sum_{i=1}^{n} \frac{x_{i}(1-x_{i})}{ax_{i}(1-x_{i}) + \epsilon_{i}} \\
  \end{split}
\end{equation}
og finner så
\begin{equation}
  \label{eq:17}
  0 = \sum_{i=1}^{n} \frac{x_{i}(1-x_{i})Y_{i}}{}
\end{equation}


\end{document}
