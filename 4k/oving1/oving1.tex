\documentclass{report}

\usepackage{amsfonts, amsmath, amssymb, amsthm}
\usepackage[margin=1.0in]{geometry}
\usepackage{hyperref}
\usepackage{float}
\usepackage{fancyhdr}
\usepackage{graphicx}
\usepackage{mathrsfs}
\usepackage{comment}

\newcommand{\Dim}{\text{dim}\;}
\newcommand{\Col}{\text{Col}\;}
\newcommand{\Row}{\text{Row}\;}
\newcommand{\Null}{\text{Null}\;}
\newcommand{\Ker}{\text{ker}\;}
\newcommand{\Image}{\text{im}\;}
\newcommand{\Sp}[1]{\text{Sp} \left\lbrace #1 \right\rbrace}
\newcommand{\M}[2]{\mathbb{#1}^{#2}}
\newcommand{\twovector}[2]{\left[ \begin{array}{c} #1 \\ #2 \\ \end{array} \right]}
\newcommand{\threevector}[3]{\left[ \begin{array}{c} #1 \\ #2 \\ #3 \\ \end{array} \right]}
\newcommand{\fourvector}[4]{\left[ \begin{array}{c} #1 \\ #2 \\ #3 \\ #4 \\ \end{array} \right]}
\newcommand{\fivevector}[5]{\left[ \begin{array}{c} #1 \\ #2 \\ #3 \\ #4 \\ #5 \\ \end{array} \right]}
\newcommand{\sixvector}[6]{\left[ \begin{array}{c} #1 \\ #2 \\ #3 \\ #4 \\ #5 \\ #6 \\ \end{array} \right]}
\newcommand{\sevenvector}[7]{\left[ \begin{array}{c} #1 \\ #2 \\ #3 \\ #4 \\ #5 \\ #6 \\ #7 \\ \end{array} \right]}
\newcommand{\eightvector}[8]{\left[ \begin{array}{c} #1 \\ #2 \\ #3 \\ #4 \\ #5 \\ #6 \\ #7 \\ #8 \\ \end{array} \right]}
\newcommand{\ninevector}[9]{\left[ \begin{array}{c} #1 \\ #2 \\ #3 \\ #4 \\ #5 \\ #6 \\ #7 \\ #8 \\ #9 \\ \end{array} \right]}
\newcommand{\nbrack}[1]{\left( #1 \right)}
\newcommand{\bbrack}[1]{\left[ #1 \right]}
\newcommand{\cbrack}[1]{\left\lbrace #1 \right\rbrace}
\newcommand{\abrack}[1]{\left< #1 \right>}
\newcommand{\linebrack}[1]{\left| #1 \right|}
\newcommand{\twomatrix}[4]{\bbrack{
    \begin{array}{cc}
      #1 & #2 \\
      #3 & #4 \\
    \end{array}
  }
}
\newcommand{\threematrix}[9]{\bbrack{
    \begin{array}{ccc}
      #1 & #2 & #3 \\
      #4 & #5 & #6 \\
      #7 & #8 & #9 \\
    \end{array}
  }
}
\newcommand{\Lplc}[1]{\mathscr{L}\bbrack{ #1 } (s)}
\newcommand{\iLplc}[1]{\mathscr{L}^{-1}\bbrack{ #1 } (t)}
\newcommand{\til}[1]{\overset{~}{#1}}

\title{Innlevering 1}
\author{Jacob Oliver Bruun}
\date{\today}

\makeatletter
\let\inserttitle\@title
\let\insertauthor\@author
\makeatother

\pagestyle{fancy}
\chead{\insertauthor}
\lhead{\inserttitle}
\rhead{\today}

\parindent 0ex


\begin{document}
\section*{6.1.1}
Har $f(t) = 2t + 8$ og skal finne Laplace transformen $\Lplc{f}$ og vet da
\begin{equation}
  \label{eq:1}
  \Lplc{f} = \Lplc{2t + 8} = 2\Lplc{t} + 8\Lplc{1} = \frac{2}{s^{2}} + \frac{8}{s}
\end{equation}


\section*{6.1.12}
Skal finne $\Lplc{f}$ og kan skrive $f(t) = \nbrack{t-1} \nbrack{1-u(t-1)} + 1$ og har også fra teorem
\begin{equation}
  \label{eq:2}
  \Lplc{g(t-a)u(t-a)} = e^{-as} \Lplc{g(t)}
\end{equation}
og kan dermed skrive
\begin{equation}
  \label{eq:3}
  \Lplc{f(t)} = \Lplc{t} - \Lplc{(t-1) u(t-1)} = \frac{1}{s^{2}} - e^{-s} \Lplc{t} = \frac{1-e^{-s}}{s^{2}}
\end{equation}


\section*{6.1.23}
Skal vise at dersom $\Lplc{f(t)} = F(t)$ og $c$ er en positiv konstant har vi $\Lplc{f(ct)} = F(s/c)/c$, prøver og sette inn i definisjonen av Laplace
\begin{equation}
  \label{eq:4}
  \Lplc{f(ct)} = \int_{0}^{\infty} e^{-st} f(ct) dt
\end{equation}
gjør en substitusjon med $u = ct$ med $du = cdt$ og får da videre
\begin{equation}
  \label{eq:5}
  \Lplc{f(ct)} = \frac{1}{c}\int_{0}^{\infty} e^{-s\frac{t}{c}} f(u) du = \frac{1}{c} \mathscr{L} [f(ct)]\nbrack{\frac{s}{c}} = \frac{F\nbrack{\frac{s}{c}}}{c}
\end{equation}
og kan dermed sette inn for å finne $\Lplc{\cos \omega t}$ og vet at $\Lplc{\cos t} = \frac{s}{\nbrack{s^{2} + 1}}$ og har dermed
\begin{equation}
  \label{eq:6}
  \Lplc{\cos \omega t} = \frac{s/c}{c \nbrack{\nbrack{s/c}^{2} + 1}}
\end{equation}


\section*{6.1.26}
Bruker delbrøkoppspaltning
\begin{equation}
\begin{split}
  \label{eq:7}
  \frac{5s + 1}{s^{2} - 25} &= \frac{A}{s-5} + \frac{B}{s+5} \\
  5s + 1 &= As + 5A + Bs - 5B
\end{split}
\end{equation}
har dermed likningssettet
\begin{equation}
  \label{eq:8}
  A + B = 5 \land 5A - 5B = 1
\end{equation}
og har dermed $A = 26/10, B = 24/10$ og har da
\begin{equation}
  \label{eq:9}
  \frac{5s + 1}{s^{2} - 25} = \frac{26}{10\nbrack{s - 5}} + \frac{24}{10\nbrack{s+5}}
\end{equation}
og finner dermed
\begin{equation}
  \iLplc{\frac{5s + 1}{s^{2} - 25}} = \frac{26}{10} \iLplc{\frac{1}{s-5}} + \frac{24}{10} \iLplc{\frac{1}{s+5}} = \frac{13}{5} e^{5t} + \frac{12}{5} e^{-5t}
\end{equation}


\section*{6.1.36}
Har $f(t) = \sinh t \cos t$ kan skrive ut
\begin{equation}
  \label{eq:10}
  f(t) = \frac{1}{2} \nbrack{e^{t} - e^{-t}} \cos t = \frac{1}{2} \nbrack{ e^{t} \cos t - e^{-t} \cos t }
\end{equation}
og har dermed
\begin{equation}
  \label{eq:11}
  \Lplc{f(t)} = \frac{1}{2} \Lplc{e^{t} \cos t} - \frac{1}{2} \Lplc{e^{-t} \cos t} = \frac{1}{2} \nbrack{ \frac{s-1}{\nbrack{s-1}^{2} + 1} - \frac{s+1}{\nbrack{s+1}^{2} + 1}}
\end{equation}



\section*{6.1.40}
Har $f(t) = \frac{4}{s^{2} - 2s - 3}$ kan bruke delbrøkoppspaltning
\begin{equation}
  \label{eq:12}
  \begin{split}
    f(t) = \frac{4}{(s-3)(s+1)} &= \frac{A}{s-3} + \frac{B}{s+1} \\
    4 &= A(s+1) + B(s-3)
  \end{split}
\end{equation}
som gir
\begin{equation}
  \label{eq:13}
  A + B = 0 \land A - 3B = 4
\end{equation}
og kan dermed se at vi må ha $B = -1, A = 1$, og finner dermed
\begin{equation}
  \label{eq:14}
  \iLplc{f} = \iLplc{\frac{1}{s+1} - \frac{1}{s-3}} = e^{3t} - e^{-t}
\end{equation}



\section*{6.2.4}
Ser på likningen $y'' + 9y = 10e^{-t}$ med $y(0) = y'(0) = 0$ og ved å Laplace transformere likningen får vi
\begin{equation}
  \label{eq:15}
  s^{2} Y + 9Y = \frac{10}{s+1} \Rightarrow Y = \frac{10}{\nbrack{s+1} \nbrack{s^{2}+9}}
\end{equation}
har dermed
\begin{equation}
  \label{eq:16}
  \begin{split}
    Y = \frac{10}{\nbrack{s+1} \nbrack{s^{2}+9}} &= \frac{A}{s+1} + \frac{Bs + C}{s^{2} + 9} \\
    10 &= As^{2} + 9A + Bs^{2} + Bs + Cs + C
  \end{split}
\end{equation}
ser en løsning $A=1, B=-1, C=1$ og har dermed
\begin{equation}
  \label{eq:17}
  Y = \frac{1}{s+1} + \frac{1}{s^{2} + 9} - \frac{s}{s^{2} + 9}
\end{equation}
kan så finne
\begin{equation}
  \label{eq:18}
  y = \iLplc{Y} = e^{-t} + \frac{1}{3} \sin 3t - \cos 3t
\end{equation}



\section*{6.2.13}
Har likningen $y' - 6y = 0$ med $y(-1) = 4$, bruker Laplace transformasjon selv om det er nærmest trivielt å løse likningen som en separabel differensial likning. Vi setter $\tilde t = t + 1$ og har da $\tilde y' - 6\tilde y = 0$ med $\tilde y (0) = 4$ og dermed
\begin{equation}
  \label{eq:20}
  s \tilde Y - 4 - 6 \tilde Y = 0 \Rightarrow \tilde Y = \frac{4}{s-6} \Rightarrow \tilde y = 4e^{6\tilde t}
\end{equation}
og har dermed
\begin{equation}
  \label{eq:21}
  y = 4e^{6(t+1)}
\end{equation}



\section*{6.3.8}
Har $f(t) = t^{2}$ for $1<t<2$ som er 0 ellers, og kan skrive funksjonen som
\begin{equation}
  \label{eq:23}
  f(t) = t^{2} \nbrack{ u(t-1) - u(t-2) }
\end{equation}
hvor $u$ er Heavyside-funksjonen, og kan finne transformen
\begin{equation}
  \label{eq:24}
  \Lplc{f(t)} = \int_{0}^{\infty} e^{-st}f(t) dt = \int_{0}^{1} e^{-st}\cdot 0 dt + \int_{1}^{2} e^{-st} t^{2} dt + \int_{2}^{\infty} e^{-st}\cdot 0 dt
\end{equation}
kan dermed bruke delvis integrasjon og dermed
\begin{equation}
  \label{eq:19}
  \begin{split}
    \Lplc{f(t)} &= \int_{1}^{2} e^{-st} t^{2} dt  \\
    &= \bbrack{ -\frac{1}{s}e^{-st} t^{2}}_{1}^{2} - \bbrack{\frac{1}{s^{2}} e^{-st}t}_{1}^{2} + \int_{1}^{2} \frac{1}{s^{2}}e^{-st} dt \\
    &= - \frac{4}{s} e^{-2s} + \frac{1}{s} e^{-s} - \frac{2}{s^{2}} e^{-2s} + \frac{1}{s^{2}} e^{-s} + \frac{1}{s^{3}} e^{-2s} - \frac{1}{s^{3}} e^{-s} \\
    &= \nbrack{ \frac{-1 + s + s^{2}}{s^{3}} } e^{-s} + \nbrack{ \frac{1 - 2s - 4s^{2}}{s^{3}} } e^{-2s}
  \end{split}
\end{equation}


\section*{6.3.15}
Ser på $\Lplc{f(t)} = e^{-2s}/s^{6}$ og kan bruke fra teorem
\begin{equation}
  \label{eq:22}
  \Lplc{g(t-a) u(t-a)} = e^{-as}\Lplc{g(t)}
\end{equation}
og kan dermed se at vi i dette tilfellet har $f(t) = g(t-a)u(t-a), a = 2, \Lplc{g(t)} = 1/s^{6}$ og kan dermed finne
\begin{equation}
  \label{eq:25}
  g(t) = \iLplc{\frac{1}{s^{6}}} = \frac{t^{5}}{5!}
\end{equation}
og har dermed
\begin{equation}
  \label{eq:26}
  f(t) = \frac{\nbrack{t-2}^{5}}{5!} u(t-2)
\end{equation}



\section*{6.3.25}
Ser på $y'' + y = 2t$ for $0<t<1$ og $y'' + y = 2$ for $t>1$ med $y(0) = 0, y'(0) = -2$, kan skrive
\begin{equation}
  \label{eq:27}
  f(t) = \nbrack{ 2t - 2 } \nbrack{ 1-u(t-1) } + 2
\end{equation}
og kan dermed ta Laplace av likningen
\begin{equation}
  \label{eq:28}
  \begin{split}
    s^{2} Y - 2 + Y &= \frac{2-2e^{-s}}{s^{2}} \\
    Y &= \frac{2-2e^{-s} + s^{2}}{s^{2}\nbrack{s^{2}+1}} = \frac{1}{s^{2}} + \frac{1}{s^{2}\nbrack{s^{2}+1}} - \frac{2e^{-s}}{s^{2} \nbrack{s^{2}+1}}
  \end{split}
\end{equation}
finner så invers Laplace av de forskjellige leddene og legger merke til at
\begin{equation}
  \label{eq:29}
  \frac{1}{s^{2} \nbrack{s^{2}+1}} = \Lplc{t} \cdot \Lplc{\sin t} = \Lplc{t*\sin t}
\end{equation}
og kan så finne
\begin{equation}
  \label{eq:30}
  \begin{split}
    t*\sin t &= \int_{0}^{t} \nbrack{t\sin \tau - \tau \sin\tau} d\tau \\
    &= -t \cos t + t + t\cos t - \sin t \\
    &= t - \sin t
  \end{split}
\end{equation}
og har dermed
\begin{equation}
  \label{eq:31}
  y = \iLplc{Y} = 2t - \sin t - 2u(t-1) \nbrack{ t - 1 - \sin \nbrack{t-1} }
\end{equation}
kan sjekke svaret og ser
\begin{equation}
  \label{eq:32}
  \begin{split}
    y' &= 2 - \cos t - 2u(t-1) - 2u(t-1) \cos (t-1) \\
    y'' &= \sin t - 2\sin(t-1)u(t-1)
  \end{split}
\end{equation}
og finner da at
\begin{equation}
  \label{eq:33}
  y + y'' = 2t - \sin t + \sin t = 2t
\end{equation}
for $0<t<1$ og
\begin{equation}
  \label{eq:34}
  y + y'' = 2t - \sin t - 2t + 2 + 2\sin (t-1) + \sin t - 2\sin(t-1) = 2
\end{equation}
men ser at $y'(0) = 1$, men dersom vi heller skriver
\begin{equation}
  \label{eq:35}
  y = 2t - 4\sin t - 2u(t-1) \nbrack{ t - 1 - \sin \nbrack{t-1} }
\end{equation}
og siden denne endringen vil bidra til en faktor på 4 i et ledd i $y''$ og $y$ som kansellerer hverandre, vil denne løsningen også være en løsning på differensiallikningen og initialbetingelsene vil være oppfylt. \\





\end{document}
