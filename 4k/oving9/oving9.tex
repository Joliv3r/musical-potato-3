\documentclass{report}

\usepackage{amsfonts, amsmath, amssymb, amsthm}
\usepackage[margin=1.0in]{geometry}
\usepackage{hyperref}
\usepackage{float}
\usepackage{fancyhdr}
\usepackage{graphicx}
\usepackage{mathrsfs}
\usepackage{comment}

\newcommand{\M}[2]{\mathbb{#1}^{#2}}
\newcommand{\twovector}[2]{\left[ \begin{array}{c} #1 \\ #2 \\ \end{array} \right]}
\newcommand{\threevector}[3]{\left[ \begin{array}{c} #1 \\ #2 \\ #3 \\ \end{array} \right]}
\newcommand{\fourvector}[4]{\left[ \begin{array}{c} #1 \\ #2 \\ #3 \\ #4 \\ \end{array} \right]}
\newcommand{\fivevector}[5]{\left[ \begin{array}{c} #1 \\ #2 \\ #3 \\ #4 \\ #5 \\ \end{array} \right]}
\newcommand{\sixvector}[6]{\left[ \begin{array}{c} #1 \\ #2 \\ #3 \\ #4 \\ #5 \\ #6 \\ \end{array} \right]}
\newcommand{\sevenvector}[7]{\left[ \begin{array}{c} #1 \\ #2 \\ #3 \\ #4 \\ #5 \\ #6 \\ #7 \\ \end{array} \right]}
\newcommand{\eightvector}[8]{\left[ \begin{array}{c} #1 \\ #2 \\ #3 \\ #4 \\ #5 \\ #6 \\ #7 \\ #8 \\ \end{array} \right]}
\newcommand{\ninevector}[9]{\left[ \begin{array}{c} #1 \\ #2 \\ #3 \\ #4 \\ #5 \\ #6 \\ #7 \\ #8 \\ #9 \\ \end{array} \right]}
\newcommand{\nbrack}[1]{\left( #1 \right)}
\newcommand{\bbrack}[1]{\left[ #1 \right]}
\newcommand{\cbrack}[1]{\left\lbrace #1 \right\rbrace}
\newcommand{\abrack}[1]{\left< #1 \right>}
\newcommand{\linebrack}[1]{\left| #1 \right|}
\newcommand{\twomatrix}[4]{\bbrack{
    \begin{array}{cc}
      #1 & #2 \\
      #3 & #4 \\
    \end{array}
  }
}
\newcommand{\threematrix}[9]{\bbrack{
    \begin{array}{ccc}
      #1 & #2 & #3 \\
      #4 & #5 & #6 \\
      #7 & #8 & #9 \\
    \end{array}
  }
}
\newcommand{\Lplc}[1]{\mathscr{L}\bbrack{ #1 } (s)}
\newcommand{\iLplc}[1]{\mathscr{L}^{-1}\bbrack{ #1 } (t)}
\newcommand{\im}{\text{Im}}
\newcommand{\re}{\text{Re}}

\title{Innlevering 7}
\author{Jacob Oliver Bruun}
\date{\today}

\makeatletter
\let\inserttitle\@title
\let\insertauthor\@author
\makeatother

\pagestyle{fancy}
\chead{\insertauthor}
\lhead{\inserttitle}
\rhead{\today}

\parindent 0ex


\begin{document}

\section*{13.7.22}
Skal finne prinsipialverdien til $z=(2i)^{2i}$ og har da
\begin{equation}
  \label{eq:1}
  z = 2^{2i} \nbrack{ e^{i\frac{\pi}{2}} }^{2i} = e^{\ln 2 \cdot 2i} e^{-\pi} = e^{-\pi}e^{i\cdot 2\ln 2}
\end{equation}
Har $2\ln 2 < \pi$ og har dermed $\arg z = 2\ln 2$.


\section*{14.1.3}
Ser på kurven
\begin{equation}
  \label{eq:2}
  \mathcal{C} : z(t) = t + 4t^{2}i, t \in \bbrack{ 0, 1 }
\end{equation}


\section*{14.1.11}
Ser på $\mathcal{C}$ fra $(-1,2)$ til $(1,4)$ og kan representere den på følgende måte
\begin{equation}
  \label{eq:3}
  z(t) = t + 3ti.
\end{equation}



\section*{14.1.20}
Ser på $\mathcal{C}$ gitt ved
\begin{equation}
  \label{eq:4}
  4(x-2)^{2} + 5(y+1)^{2} = 20 \Rightarrow \frac{1}{5}(x-2)^{2} + \frac{1}{4}(y+1)^{2} = 1
\end{equation}
kan da skrive
\begin{equation}
  \label{eq:5}
  x-2 = \sqrt{5} \cos t, \;\;\; y+1 = 2 \sin t
\end{equation}
og har da ved $z = x + iy$
\begin{equation}
  \label{eq:6}
  z(t) = \sqrt{2} \sin t + 2 + i\nbrack{ 2\cos t - 1 }
\end{equation}


\section*{14.1.22}
Ser på integralet
\begin{equation}
  \label{eq:7}
  \int_{\mathcal{C}} \re z dz, \;\;\;\; \mathcal{C} : y = 1 + \frac{1}{2}(x-1)^{2},
\end{equation}


\section*{14.1.22}
\section*{14.1.26}
\section*{14.1.29}
\section*{14.2.4}
\section*{14.2.13}
\section*{14.2.22}
\section*{14.2.23}
\section*{14.2.28}

\end{document}
