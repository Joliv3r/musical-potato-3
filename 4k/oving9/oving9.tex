\documentclass{report}

\usepackage{amsfonts, amsmath, amssymb, amsthm}
\usepackage[margin=1.0in]{geometry}
\usepackage{hyperref}
\usepackage{float}
\usepackage{fancyhdr}
\usepackage{graphicx}
\usepackage{mathrsfs}
\usepackage{comment}

\newcommand{\M}[2]{\mathbb{#1}^{#2}}
\newcommand{\twovector}[2]{\left[ \begin{array}{c} #1 \\ #2 \\ \end{array} \right]}
\newcommand{\threevector}[3]{\left[ \begin{array}{c} #1 \\ #2 \\ #3 \\ \end{array} \right]}
\newcommand{\fourvector}[4]{\left[ \begin{array}{c} #1 \\ #2 \\ #3 \\ #4 \\ \end{array} \right]}
\newcommand{\fivevector}[5]{\left[ \begin{array}{c} #1 \\ #2 \\ #3 \\ #4 \\ #5 \\ \end{array} \right]}
\newcommand{\sixvector}[6]{\left[ \begin{array}{c} #1 \\ #2 \\ #3 \\ #4 \\ #5 \\ #6 \\ \end{array} \right]}
\newcommand{\sevenvector}[7]{\left[ \begin{array}{c} #1 \\ #2 \\ #3 \\ #4 \\ #5 \\ #6 \\ #7 \\ \end{array} \right]}
\newcommand{\eightvector}[8]{\left[ \begin{array}{c} #1 \\ #2 \\ #3 \\ #4 \\ #5 \\ #6 \\ #7 \\ #8 \\ \end{array} \right]}
\newcommand{\ninevector}[9]{\left[ \begin{array}{c} #1 \\ #2 \\ #3 \\ #4 \\ #5 \\ #6 \\ #7 \\ #8 \\ #9 \\ \end{array} \right]}
\newcommand{\nbrack}[1]{\left( #1 \right)}
\newcommand{\bbrack}[1]{\left[ #1 \right]}
\newcommand{\cbrack}[1]{\left\lbrace #1 \right\rbrace}
\newcommand{\abrack}[1]{\left< #1 \right>}
\newcommand{\linebrack}[1]{\left| #1 \right|}
\newcommand{\twomatrix}[4]{\bbrack{
    \begin{array}{cc}
      #1 & #2 \\
      #3 & #4 \\
    \end{array}
  }
}
\newcommand{\threematrix}[9]{\bbrack{
    \begin{array}{ccc}
      #1 & #2 & #3 \\
      #4 & #5 & #6 \\
      #7 & #8 & #9 \\
    \end{array}
  }
}
\newcommand{\Lplc}[1]{\mathscr{L}\bbrack{ #1 } (s)}
\newcommand{\iLplc}[1]{\mathscr{L}^{-1}\bbrack{ #1 } (t)}
\newcommand{\im}{\text{Im}}
\newcommand{\re}{\text{Re}}

\title{Innlevering 7}
\author{Jacob Oliver Bruun}
\date{\today}

\makeatletter
\let\inserttitle\@title
\let\insertauthor\@author
\makeatother

\pagestyle{fancy}
\chead{\insertauthor}
\lhead{\inserttitle}
\rhead{\today}

\parindent 0ex


\begin{document}

\section*{13.7.22}
Skal finne prinsipialverdien til $z=(2i)^{2i}$ og har da
\begin{equation}
  \label{eq:1}
  z = 2^{2i} \nbrack{ e^{i\frac{\pi}{2}} }^{2i} = e^{\ln 2 \cdot 2i} e^{-\pi} = e^{-\pi}e^{i\cdot 2\ln 2}
\end{equation}
Har $2\ln 2 < \pi$ og har dermed $\arg z = 2\ln 2$.


\section*{14.1.3}
Ser på kurven
\begin{equation}
  \label{eq:2}
  \mathcal{C} : z(t) = t + 4t^{2}i, t \in \bbrack{ 0, 1 }
\end{equation}


\section*{14.1.11}
Ser på $\mathcal{C}$ fra $(-1,2)$ til $(1,4)$ og kan representere den på følgende måte
\begin{equation}
  \label{eq:3}
  z(t) = t + 3ti.
\end{equation}



\section*{14.1.20}
Ser på $\mathcal{C}$ gitt ved
\begin{equation}
  \label{eq:4}
  4(x-2)^{2} + 5(y+1)^{2} = 20 \Rightarrow \frac{1}{5}(x-2)^{2} + \frac{1}{4}(y+1)^{2} = 1
\end{equation}
kan da skrive
\begin{equation}
  \label{eq:5}
  x-2 = \sqrt{5} \cos t, \;\;\; y+1 = 2 \sin t
\end{equation}
og har da ved $z = x + iy$
\begin{equation}
  \label{eq:6}
  z(t) = \sqrt{2} \sin t + 2 + i\nbrack{ 2\cos t - 1 }
\end{equation}


\section*{14.1.22}
Ser på integralet
\begin{equation}
  \label{eq:7}
  \int_{\mathcal{C}} \re z dz, \;\;\;\; \mathcal{C} : y = 1 + \frac{1}{2}(x-1)^{2}
\end{equation}
fra $1+i$ til $3+3i$ og kan parametrisere kurven ved
\begin{equation}
  \label{eq:8}
  z(t) = t + i\nbrack{ 1 + \frac{1}{2}(t-1)^{2} }, t \in (1, 3)
\end{equation}
og har da
\begin{equation}
  \label{eq:9}
  z'(t) = 1 + ti - i
\end{equation}
som da gir oss
\begin{equation}
  \label{eq:10}
  \int_{\mathcal{C}} \re z dz = \int_{1}^{3} \bbrack{ \re z(t) } z'(t) dt = \int_{1}^{3} \bbrack{ t + t^{2}i - it } dt = 4(1-i) + \frac{26}{3}i = 4 + \frac{14}{3} i
\end{equation}


\section*{14.1.26}
Ser på integralet
\begin{equation}
  \label{eq:11}
  \int_{\mathcal{C}} (z + z^{-1}) dz, \;\;\;\; \mathcal{C} : z(t) = e^{it}, t\in \left[ 0, 2\pi \right)
\end{equation}
kan dele opp og ser fordi $z$ er analytisk på hele $\M{C}{}$ så har vi
\begin{equation}
  \label{eq:12}
  \int_{\mathcal{C}} (z + z^{-1}) dz = \int_{\mathcal{C}} z dz + \int_{\mathcal{C}} z^{-1}dz = \int_{\mathcal{C}} z^{-1} dz
\end{equation}
kan så finne
\begin{equation}
  \label{eq:13}
  \int_{\mathcal{C}} z^{-1} dz = \int_{0}^{2\pi} \frac{ie^{it}}{e^{it}}dt = 2\pi i
\end{equation}


\section*{14.1.29}
Ser på integralet
\begin{equation}
  \label{eq:14}
  \int_{\mathcal{C}} \im z^{2} dz = \int_{\mathcal{C}_{1}} \im z^{2} dz + \int_{\mathcal{C}_{2}} \im z^{2} dz + \int_{\mathcal{C}_{3}} \im z^{2} dz
\end{equation}
med $\mathcal{C}$ gitt av trekanten gitt av punktene $0, 1, i$ og $\mathcal{C}_{1}$ linja mellom $0, 1$ og $\mathcal{C}_{2}$ linja mellom $1, i$ og $\mathcal{C}_{3}$ linja mellom $i, 0$. Vet at $\im z^{2} = 0 \;\; \forall z \in \M{R}{}$ og har dermed integralet over $\mathcal{C}_{1}$ lik 0. Har da parameterfremstillingene
\begin{equation}
  \label{eq:15}
  \begin{array}{lll}
    \mathcal{C}_{2} :& z(t) = 1 - t + it & , t\in \bbrack{0, 1} \\
    \mathcal{C}_{3} :& z(t) = it & , t\in \bbrack{0, 1}
  \end{array}
\end{equation}
og finner så
\begin{equation}
  \label{eq:16}
  \begin{split}
    \int_{\mathcal{C}} \im z^{2} dz &= \int_{0}^{1} (i-1) \im \bbrack{ (1 - t + it)^{2} } dt + \int_{0}^{1} i \im \bbrack{ (it)^{2} } dt \\
                                    &= \int_{0}^{1} (i-1) \bbrack{ 2t - 2t^{2} } dt \\
                                    &= (2i - 2) \int_{0}^{1} \bbrack{ t - t^{2}} dt \\
                                    &= (2i - 2) \bbrack{ \frac{1}{2} - \frac{1}{3} } \\
                                    &= \frac{1}{3}i - \frac{1}{3}
  \end{split}
\end{equation}


\section*{14.2.4}
Har at en funksjon, la oss kalle den $f(z)$, har følgende egenskaper
\begin{equation}
  \label{eq:17}
  \int_{R_{1}} f(z)dz = 2, \;\;\;\; \int_{R_{3}} f(z)dz = 6
\end{equation}
hvor $R_{r}$ er sirkelen med sentrum i origo og radius $r$, vil se om funksjonen er analytisk i annulusen $1 < |z| < 3$ og kaller dette området $R$. Vet at for en analytisk funksjon på hele domenet $D$ og med en lukket kurve $\mathcal C \subset D$ har vi
\begin{equation}
  \label{eq:18}
  \oint_{\mathcal{C}} f(z) dz = 0
\end{equation}
siden vi vet at
\begin{equation}
  \label{eq:19}
  \oint_{\partial R} f(z) dz = \int_{R_{1}} f(z) dz - \int_{R_{3}} f(z) dz = -4 \neq 0
\end{equation}
så er ikke $f(z)$ analytisk på hele $R$.


\section*{14.2.13}
Ser på integralet over enhetssirkelen mot klokka
\begin{equation}
  \label{eq:20}
  \oint_{\mathcal{C}} \nbrack{z^{2} - 1.2}^{-1} dz
\end{equation}
og siden dette er integralet over en lukket kurve i et enkeltsammenhengende domene hvor integranden er analytisk har vi
\begin{equation}
  \label{eq:22}
  \oint_{\mathcal{C}} \nbrack{z^{2} - 1.2}^{-1} dz = 0
\end{equation}

\section*{14.2.22}
Ser på integralet
\begin{equation}
  \label{eq:23}
  \oint_{\mathcal{C}} \re z dz = \int_{\mathcal{C}_{1} \cup \mathcal{C}_{2}}  \re z dz,
  \;\;\;\; \mathcal{C}_{1} : z(t) = e^{it}, t\in \bbrack{ 0, \pi }, \mathcal{C}_{2} : z(t) = t, t \in \bbrack{-1, 1}
\end{equation}
og har dermed
\begin{equation}
  \label{eq:24}
  \int_{\mathcal{C}_{1}} \re z dz = \int_{0}^{\pi} \re \bbrack{ e^{it} } ie^{it}dt = i\int_{0}^{\pi} e^{it} dt = e^{i\pi} - 1 = -2
\end{equation}
og
\begin{equation}
  \label{eq:25}
  \int_{\mathcal{C}_{2}} \re z dz = \int_{-1}^{1} t dt = 0
\end{equation}
og har dermed
\begin{equation}
  \label{eq:26}
  \oint_{\mathcal{C}} \re z dz = -2
\end{equation}


\section*{14.2.23}
Ser på integralet
\begin{equation}
  \label{eq:21}
  \oint_{\mathcal{C}} \frac{2z-1}{z^{2} - z} dz
\end{equation}
kaller integranden $f(z)$ og ser at for $z=0,1$ er funksjonen ikke analytisk. Ser dermed på kurvene $\mathcal{C}_{1}, \mathcal{C}_{2}$ som er en sirkel med radius 1 rundt disse punktene og har da
\begin{equation}
  \label{eq:31}
  \oint_{\mathcal{C}} f(z) dz = \oint_{\mathcal{C}_{1}} f(z) dz + \oint_{\mathcal{C}_{2}} f(z) dz
\end{equation}
har da parameterfremstillingene
\begin{equation}
  \label{eq:33}
  \begin{array}{lll}
    \mathcal{C}_{1}: & z(t) = e^{it} & t \in \left[ 0, 2\pi \right) \\
    \mathcal{C}_{2}: & z(t) = 2 + e^{it} & t \in \left[ 0, 2\pi \right)
  \end{array}
\end{equation}
og kan dermed skrive
\begin{equation}
  \label{eq:32}
  \begin{split}
    \oint_{\mathcal{C}} f(z) dz
    &= \int_{0}^{2\pi} \frac{2e^{it} - 1}{e^{it} \nbrack{ e^{it} - 1 }} ie^{it} dt + \int_{0}^{2\pi} \frac{ 4 + 2e^{it} - 1 }{ \nbrack{ 2+e^{it} } \nbrack{ 2 + e^{it} - 1 } } ie^{it} dt \\
    &= i\int_{0}^{2\pi} \bbrack{ \frac{2e^{it} - 1}{e^{it} - 1} + \frac{ 3e^{it} + 2e^{2it} }{ \nbrack{e^{it}+2} \nbrack{e^{it}+1} } }dt \\
    &= i \int_{0}^{2\pi} \frac{ \nbrack{2e^{it}-1} \nbrack{e^{2it} + 3e^{it} + 2} + 2e^{3it} + e^{2it} - 3e^{it} }{ \nbrack{e^{it}-1} \nbrack{e^{it}+1} \nbrack{e^{it}+2} } dt \\
    &= i \int_{0}^{2\pi} \frac{ 4e^{3it} + 6e^{2it} - 2e^{it} - 2 }{\nbrack{e^{it}-1} \nbrack{e^{it}+1} \nbrack{e^{it}+2}} dt
  \end{split}
\end{equation}


\section*{14.2.28}
Ser på integralet
\begin{equation}
  \label{eq:27}
  \oint_{\mathcal{C}} \frac{\tan \frac{z}{2}}{16z^{4} - 81} dz
\end{equation}
hvor $\mathcal{C}$ er kvadretet definert av hjørnene $\pm 1, \pm i$, og ser at integranden ikke er analytisk i
\begin{equation}
  \label{eq:28}
  z^{4} = \frac{81}{16} \Rightarrow |z| = \frac{3}{2}
\end{equation}
den er heller ikke analytisk for
\begin{equation}
  \label{eq:29}
  \cos z = 0 \Rightarrow z = \frac{\pi}{2} + \pi k, k \in \M{Z}{}
\end{equation}
vet at på kvadratets maksimale distanse fra origo er 1 og funksjonen er analytisk innenfor dette området, og har dermed
\begin{equation}
  \label{eq:30}
  \oint_{\mathcal{C}} \frac{\tan \frac{z}{2}}{16z^{4} - 81} dz = 0
\end{equation}


\end{document}
