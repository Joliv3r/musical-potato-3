\documentclass{report}

\usepackage{amsfonts, amsmath, amssymb, amsthm}
\usepackage[margin=1.0in]{geometry}
\usepackage{hyperref}
\usepackage{float}
\usepackage{fancyhdr}
\usepackage{graphicx}
\usepackage{mathrsfs}
\usepackage{comment}

\newcommand{\M}[2]{\mathbb{#1}^{#2}}
\newcommand{\twovector}[2]{\left[ \begin{array}{c} #1 \\ #2 \\ \end{array} \right]}
\newcommand{\threevector}[3]{\left[ \begin{array}{c} #1 \\ #2 \\ #3 \\ \end{array} \right]}
\newcommand{\fourvector}[4]{\left[ \begin{array}{c} #1 \\ #2 \\ #3 \\ #4 \\ \end{array} \right]}
\newcommand{\fivevector}[5]{\left[ \begin{array}{c} #1 \\ #2 \\ #3 \\ #4 \\ #5 \\ \end{array} \right]}
\newcommand{\sixvector}[6]{\left[ \begin{array}{c} #1 \\ #2 \\ #3 \\ #4 \\ #5 \\ #6 \\ \end{array} \right]}
\newcommand{\sevenvector}[7]{\left[ \begin{array}{c} #1 \\ #2 \\ #3 \\ #4 \\ #5 \\ #6 \\ #7 \\ \end{array} \right]}
\newcommand{\eightvector}[8]{\left[ \begin{array}{c} #1 \\ #2 \\ #3 \\ #4 \\ #5 \\ #6 \\ #7 \\ #8 \\ \end{array} \right]}
\newcommand{\ninevector}[9]{\left[ \begin{array}{c} #1 \\ #2 \\ #3 \\ #4 \\ #5 \\ #6 \\ #7 \\ #8 \\ #9 \\ \end{array} \right]}
\newcommand{\nbrack}[1]{\left( #1 \right)}
\newcommand{\bbrack}[1]{\left[ #1 \right]}
\newcommand{\cbrack}[1]{\left\lbrace #1 \right\rbrace}
\newcommand{\abrack}[1]{\left< #1 \right>}
\newcommand{\linebrack}[1]{\left| #1 \right|}
\newcommand{\twomatrix}[4]{\bbrack{
    \begin{array}{cc}
      #1 & #2 \\
      #3 & #4 \\
    \end{array}
  }
}
\newcommand{\threematrix}[9]{\bbrack{
    \begin{array}{ccc}
      #1 & #2 & #3 \\
      #4 & #5 & #6 \\
      #7 & #8 & #9 \\
    \end{array}
  }
}
\newcommand{\Lplc}[1]{\mathscr{L}\bbrack{ #1 } (s)}
\newcommand{\iLplc}[1]{\mathscr{L}^{-1}\bbrack{ #1 } (t)}
\newcommand{\til}[1]{\overset{~}{#1}}
\newcommand{\fft}[1]{\mathcal{F} \bbrack{ #1 }}
\newcommand{\ifft}[1]{\mathcal{F}^{-1} \bbrack{ #1 }}

\title{Innlevering 6}
\author{Jacob Oliver Bruun}
\date{\today}

\makeatletter
\let\inserttitle\@title
\let\insertauthor\@author
\makeatother

\pagestyle{fancy}
\chead{\insertauthor}
\lhead{\inserttitle}
\rhead{\today}

\parindent 0ex

\begin{document}

\section*{12.6.11}
Har $u_{x}(0, t) = 0 = u_{x}(L, t)$ og $u(x, t) = f(x)$ og løser varmelikningen med hensyn på dette
\begin{equation}
  \label{eq:1}
  \partial_{t} u = c^{2} \partial_{x}^{2} u
\end{equation}
og kan anta $u = F(x) \cdot G(t)$ og finner da
\begin{equation}
  \label{eq:2}
  F(x) G'(t) = c^{2}F''(x) G(x) \Rightarrow \frac{G'(t)}{c^{2}G(t)} = \frac{F''(x)}{F(x)} = \kappa
\end{equation}
som da gir likningene
\begin{equation}
  \label{eq:3}
  \left\lbrace
  \begin{array}{l}
    F'' - \kappa F = 0 \\
    G' - c^{2} \kappa G = 0
  \end{array}
  \right.
\end{equation}
ser på $F$ og ved initialbetingelsene
\begin{equation}
  \label{eq:5}
  F'(0) = 0 = F'(L)
\end{equation}
finner at for $\kappa = 0$ får vi $F(x) = ax + b$, men fordi vi har $F'(L) = a = 0$ får vi
\begin{equation}
  \label{eq:4}
  F(x) = b
\end{equation}
og kan så se på $\kappa = q^{2}$ og har da
\begin{equation}
  \label{eq:6}
  F(x) = Ae^{qx} + Be^{-qx}
\end{equation}
fra initial betingelsene har vi
\begin{equation}
  \label{eq:7}
  F'(0) = qA - qB = 0 = qAe^{qL} - qBe^{-qL} = F'(L)
\end{equation}
og har da $A=B=0$ og dermed $F = 0$ som ikke er interessant, ser videre på $\kappa = -q^{2}$ og har da
\begin{equation}
  \label{eq:8}
  F(x) = C\cos qx + D\sin qx
\end{equation}
og har da ved initialbetingelsene
\begin{equation}
  \label{eq:9}
  F'(0) = qD = 0 \Rightarrow F'(L) = -qC\sin qL = 0 \Rightarrow q = \frac{n\pi}{L}, \;\; n \in \M{Z}{}
\end{equation}
siden vi ikke ser på løsningene hvor $D = 0$ siden det vil føre til $F = 0$ og kan da se på $G$
\begin{equation}
  \label{eq:10}
  G' + c^{2} q^{2}G = 0 \Rightarrow G = \alpha e^{-c^{2}q^{2}t}
\end{equation}
og har da
\begin{equation}
  \label{eq:11}
  u_{n}(x, t) = \beta_{n} e^{-\nbrack{ \frac{cn\pi}{L} }^{2}t} \cos \frac{n\pi x}{L}
\end{equation}
og har dermed
\begin{equation}
  \label{eq:12}
  u(x, t) = \beta_{0} + \sum_{n = 1}^{\infty} \beta_{n} e^{-\nbrack{ \frac{cn\pi}{L} }^{2}t} \cos \frac{n\pi x}{L}
\end{equation}
kan så sette $t=0$ og har da
\begin{equation}
  \label{eq:13}
  u(x, 0) = \beta_{0} + \sum_{n = 1}^{\infty} \beta_{n} \cos \frac{n\pi x}{L} = f(x)
\end{equation}
ser at dette er en Fourier-cosinusrekke til $f(x)$ og har da
\begin{equation}
  \label{eq:14}
  \beta_{0} = \frac{1}{L} \int_{0}^{L} f(x) dx, \;\;\; \beta_{n} = \frac{2}{L} \int_{0}^{L} f(x) \cos \frac{n\pi x}{L} dx
\end{equation}



\section*{12.6.12}
Ser på $f(x) = x$ med $L=\pi, c=1$ og setter inn i uttrykket vi fant i forrige oppgave
\begin{equation}
  \label{eq:15}
  \beta_{0} = \frac{1}{\pi} \int_{0}^{\pi} x dx = \frac{\pi}{2}
\end{equation}
og
\begin{equation}
  \label{eq:16}
  \beta_{n} = \frac{2}{\pi} \int_{0}^{\pi} x\cos nx dx = \frac{2}{\pi} \frac{\cos n\pi - 1}{n^{2}} = \frac{ 2\bbrack{ (-1)^{n} - 1 }}{\pi n^{2}}
\end{equation}
og har da
\begin{equation}
  \label{eq:17}
  u(x, t) = \frac{\pi}{2} + \sum_{n=1}^{\infty} \frac{2\bbrack{ (-1)^{n} - 1 }}{\pi n^{2}} \cos (nx) e^{-n^{2}t}
\end{equation}


\section*{12.6.14}
Ser på $f(x) = \cos 2x$ med $L=\pi, c=1$ og setter inn i uttrykket
\begin{equation}
  \label{eq:18}
  \beta_{0} = \frac{1}{\pi} \int_{0}^{\pi} \cos 2x dx = 0
\end{equation}
og
\begin{equation}
  \label{eq:19}
  \begin{split}
    \beta_{n} &= \frac{2}{\pi} \int_{0}^{\pi} \cos 2x \cos nx dx \\
              &= \frac{1}{2\pi} \int_{0}^{\pi} \nbrack{ e^{2ix} + e^{-2ix} } \nbrack{ e^{inx} + e^{-inx} } dx \\
              &= \frac{1}{2\pi} \int_{0}^{\pi} \bbrack{ e^{(2+n) ix} + e^{(2-n) ix} + e^{-(2+n) ix} + e^{-(2-n) ix} } dx \\
              &= \frac{2}{\pi} \int_{0}^{\pi} \cbrack{ \cos \bbrack{ (2+n) x } + \cos \bbrack{ (2-n) x } } dx \\
              &= \frac{2}{\pi} \bbrack{ \frac{\sin \bbrack{ (2+n) x }}{2+n} + \frac{\sin \bbrack{ (2-n) x }}{2-n} }_{0}^{\pi} \\
              &= 0
  \end{split}
\end{equation}
og kan dermed konkludere med at $u(x, t) = 0$



\section*{12.6.16}
Har likningen $u_{t} = c^{2}u_{xx} + H$ med $H > 0$ og har $L=\pi$ med $u(0) = 0 = u(L)$ og setter
\begin{equation}
  \label{eq:20}
  u = v - \frac{Hx(x-\pi)}{2c^{2}}
\end{equation}
og finner da
\begin{equation}
  \label{eq:21}
  v_{t} = v_{xx}
\end{equation}
kan så sette $v = F(x)G(t)$ og finner
\begin{equation}
  \label{eq:22}
  G'(t)F(x) = F''(x)G(t) \Rightarrow \frac{G'(t)}{G(t)} = \frac{F''(x)}{F(x)} = \kappa
\end{equation}
vet at kun $\kappa = -q^{2}$ vil gi $F\neq 0$ og kan dermed se dette tilfellet
\begin{equation}
  \label{eq:23}
  F(x) = A\sin qx + B\cos qx \Rightarrow B=0, \;\; q = n, n \in \M{Z}{}
\end{equation}
og dermed har vi
\begin{equation}
  \label{eq:24}
  G(t) = \alpha e^{-n^{2}t}
\end{equation}
og har dermed
\begin{equation}
  \label{eq:25}
  v = \sum_{n=1}^{\infty} \beta_{n} e^{-n^{2}t} \sin nx
\end{equation}
med
\begin{equation}
  \label{eq:26}
  \beta_{n} = \frac{2}{\pi} \int_{0}^{\pi} f(x) \sin nx dx
\end{equation}
for en funksjon $f(x) = v(x, 0)$ og kan sette dette inn i den opprinnelige funksjonen
\begin{equation}
  \label{eq:27}
  u(x, t) = \sum_{n=1}^{\infty} \beta_{n} e^{-n^{2}t} \sin nx + \frac{Hx(x-\pi)}{2c^{2}}
\end{equation}
med
\begin{equation}
  \label{eq:28}
  \beta_{n} = \frac{2}{\pi} \int_{0}^{\pi} f(x) \sin nx dx, \;\;\;
  f(x) = u(x, 0) + \frac{Hx(x-\pi)}{2c^{2}}
\end{equation}



\section*{12.6.21}
Har en boks med $u(x, 0) = 0, u(x, a) = 25, u(0, y) = 0, u(a, y) = 0$ med $a=24$ og kan bruke varmelikningen uten leddet for tidsavhengighet siden $u_{t} = 0$
\begin{equation}
  \label{eq:29}
  u_{xx} =  -u_{yy}
\end{equation}
og skriver
\begin{equation}
  \label{eq:30}
  u = F(x) G(y) \Rightarrow \frac{F''(x)}{F(x)} = -\frac{G''(y)}{G(y)} = \kappa
\end{equation}
og har da likningene
\begin{equation}
  \label{eq:31}
  \left\lbrace
  \begin{array}{l}
    F'' - \kappa F = 0 \\
    G'' + \kappa G = 0
  \end{array}
  \right.
\end{equation}
må ha $\kappa = -q^{2}$ for $F$ og får da med $F(0) = 0 = F(a)$
\begin{equation}
  \label{eq:32}
  F = A\sin qx + B\cos qx \Rightarrow B=0, \;\; q = \frac{n\pi}{a}
\end{equation}
og kan fra dette også finne $G$
\begin{equation}
  \label{eq:33}
  G'' - q^{2} G = 0 \Rightarrow G = Ce^{qy} + De^{-qy}
\end{equation}
og så
\begin{equation}
  \label{eq:34}
  G(0) = C + D = 0 \Rightarrow C = -D
\end{equation}
og har da
\begin{equation}
  \label{eq:37}
  G(y) = 2C \sinh \frac{n\pi y}{a}
\end{equation}
og kan dermed finne
\begin{equation}
  \label{eq:36}
  u = \sum_{n=1}^{\infty} C_{n} \sinh \frac{n\pi y}{a} \sin \frac{n\pi x}{a}
\end{equation}
vet at $u(x, a) = 25$ og kan dermed Fourier-sinusrekka til denne funksjonen fra 0 til $a$ og har da
\begin{equation}
  \label{eq:39}
  C_{n} \sinh n\pi = \frac{2}{a} \int_{0}^{a} 25 \sin \frac{n\pi x}{a} dx
  = \frac{50}{n\pi} \bbrack{ 1 - \cos n\pi } = \frac{100}{n\pi}, n = 2k-1, k\in \M{N}{}
\end{equation}
og har da
\begin{equation}
  \label{eq:40}
  C_{n} = \frac{100}{n\pi \sinh n \pi}
\end{equation}
og har dermed
\begin{equation}
  \label{eq:41}
  u(x, y) = \sum_{n=1}^{\infty} \frac{100}{n\pi \sinh n\pi} \sinh \frac{n\pi y}{24} \sin\frac{n\pi x}{24}
\end{equation}



\section*{12.7.1}
Har
\begin{equation}
  \label{eq:35}
  2 u_{x} + 3u_{t} = 0, \;\; u(x, 0) = f(x)
\end{equation}
skriver om til
\begin{equation}
  \label{eq:51}
  u_{t} + cu_{x} = 0, c = \frac{2}{3}
\end{equation}
gjør variabelskiftet $\xi = x + ct, \eta = x - ct$ og har dermed
\begin{equation}
  \label{eq:52}
  \frac{\partial}{\partial x}
  = \frac{\partial \xi}{\partial x} \frac{\partial}{\partial \xi} + \frac{\partial \eta}{\partial x} \frac{\partial}{\partial \eta}
  = \frac{\partial}{\partial \xi} + \frac{\partial}{\partial \eta}
\end{equation}
og
\begin{equation}
  \label{eq:53}
  \frac{\partial}{\partial t}
  = \frac{\partial \xi}{\partial t} \frac{\partial}{\partial \xi} + \frac{\partial \eta}{\partial t} \frac{\partial}{\partial \eta}
  = c\bbrack{ \frac{\partial}{\partial \xi} - \frac{\partial}{\partial\eta} }
\end{equation}
og har dermed likningen
\begin{equation}
  \label{eq:54}
  \begin{split}
    cu_{\xi} - cu_{\eta} + cu_{\xi} + cu_{\eta} &= 0 \\
    2cu_{\xi} &= 0
  \end{split}
\end{equation}
som da gir oss $u = F(\eta) = F(x - ct)$ ser at hvis vi setter $t=0$ får vi
\begin{equation}
  \label{eq:56}
  u(x, 0) = F(x) = f(x)
\end{equation}
og har dermed
\begin{equation}
  \label{eq:57}
  u(x, 0) = f\nbrack{ x - \frac{2}{3}t }
\end{equation}



\section*{12.7.2}
Har
\begin{equation}
  \label{eq:58}
  2tu_{x} + 3u_{t} = 0, \;\; u(x, 0) = f(x)
\end{equation}
ser på konstante linjer i planet som da har formen $x = x(r), t = t(r)$ og har da ved kjerneregelen
\begin{equation}
  \label{eq:59}
  \frac{du}{dr} = \frac{\partial u}{\partial t} \frac{dt}{dr} + \frac{\partial u}{\partial x}\frac{dx}{dr} = 0
\end{equation}
og har da
\begin{equation}
  \label{eq:60}
  \frac{dt}{dr} = 1, \frac{dx}{dr} = \frac{2t}{3}
\end{equation}
ved å multiplisere den opprinnelige likningen med $1/t$ får vi
\begin{equation}
  \label{eq:62}
  \frac{dt}{dr} = \frac{1}{t}, \frac{dx}{dr} = \frac{2}{3}
\end{equation}
siden $r=0$ ikke må være definert ser vi bort ifra den ene vilkårlige variabelen og har
\begin{equation}
  \label{eq:63}
  r = \frac{t^{2}}{2}, \;\; r = \frac{3}{2}x + C
\end{equation}
som kan skrives om til
\begin{equation}
  \label{eq:64}
  r = \frac{t^{2}}{2}, C = \frac{t^{2}}{2} - \frac{3}{2}x
\end{equation}
og siden $u$ er konstant på kurva så har vi
\begin{equation}
  \label{eq:65}
  u = F\nbrack{ \frac{t^{2}}{2} - \frac{3}{2}x }
\end{equation}
ved å sette $t=0$ har vi
\begin{equation}
  \label{eq:66}
  F\nbrack{ -\frac{3}{2}x } = f(x) \Rightarrow F(x) = f\nbrack{-\frac{2}{3} x}
\end{equation}
og har dermed løsningen
\begin{equation}
  \label{eq:67}
  u(x, t) = f\nbrack{ x - \frac{t^{2}}{3} }
\end{equation}



\section*{12.7.3}
Har
\begin{equation}
  \label{eq:68}
  tu_{xx} = u_{t}, \;\; u(x, 0) = f(x)
\end{equation}
med $x\in\M{R}{}, t \geq 0$ og kan bruke Fouriertransformen
\begin{equation}
  \label{eq:69}
  -tw^{2} \hat{u} = \hat{u}_{t} \Rightarrow \hat{u} = Ae^{-w^{2}t^{2}}
\end{equation}
ved å sette $t=0$ får vi $A = \hat{u}(x, 0)$ og har dermed
\begin{equation}
  \label{eq:70}
  \begin{split}
    u &= \ifft{ \hat{f}(x) e^{-w^{2}t^{2}} } \\
      &= \frac{1}{\sqrt{2\pi}} \bbrack{ f(x) * e^{-w^{2}t^{2}} } \\
      &= \frac{1}{\sqrt{2\pi}} \int_{-\infty}^{\infty} f(x-s) e^{-s^{2}t^{2}} ds \\
  \end{split}
\end{equation}








\end{document}
