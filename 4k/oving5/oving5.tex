\documentclass{report}

\usepackage{amsfonts, amsmath, amssymb, amsthm}
\usepackage[margin=1.0in]{geometry}
\usepackage{hyperref}
\usepackage{float}
\usepackage{fancyhdr}
\usepackage{graphicx}
\usepackage{mathrsfs}
\usepackage{comment}

\newcommand{\M}[2]{\mathbb{#1}^{#2}}
\newcommand{\twovector}[2]{\left[ \begin{array}{c} #1 \\ #2 \\ \end{array} \right]}
\newcommand{\threevector}[3]{\left[ \begin{array}{c} #1 \\ #2 \\ #3 \\ \end{array} \right]}
\newcommand{\fourvector}[4]{\left[ \begin{array}{c} #1 \\ #2 \\ #3 \\ #4 \\ \end{array} \right]}
\newcommand{\fivevector}[5]{\left[ \begin{array}{c} #1 \\ #2 \\ #3 \\ #4 \\ #5 \\ \end{array} \right]}
\newcommand{\sixvector}[6]{\left[ \begin{array}{c} #1 \\ #2 \\ #3 \\ #4 \\ #5 \\ #6 \\ \end{array} \right]}
\newcommand{\sevenvector}[7]{\left[ \begin{array}{c} #1 \\ #2 \\ #3 \\ #4 \\ #5 \\ #6 \\ #7 \\ \end{array} \right]}
\newcommand{\eightvector}[8]{\left[ \begin{array}{c} #1 \\ #2 \\ #3 \\ #4 \\ #5 \\ #6 \\ #7 \\ #8 \\ \end{array} \right]}
\newcommand{\ninevector}[9]{\left[ \begin{array}{c} #1 \\ #2 \\ #3 \\ #4 \\ #5 \\ #6 \\ #7 \\ #8 \\ #9 \\ \end{array} \right]}
\newcommand{\nbrack}[1]{\left( #1 \right)}
\newcommand{\bbrack}[1]{\left[ #1 \right]}
\newcommand{\cbrack}[1]{\left\lbrace #1 \right\rbrace}
\newcommand{\abrack}[1]{\left< #1 \right>}
\newcommand{\linebrack}[1]{\left| #1 \right|}
\newcommand{\twomatrix}[4]{\bbrack{
    \begin{array}{cc}
      #1 & #2 \\
      #3 & #4 \\
    \end{array}
  }
}
\newcommand{\threematrix}[9]{\bbrack{
    \begin{array}{ccc}
      #1 & #2 & #3 \\
      #4 & #5 & #6 \\
      #7 & #8 & #9 \\
    \end{array}
  }
}
\newcommand{\Lplc}[1]{\mathscr{L}\bbrack{ #1 } (s)}
\newcommand{\iLplc}[1]{\mathscr{L}^{-1}\bbrack{ #1 } (t)}
\newcommand{\til}[1]{\overset{~}{#1}}

\title{Innlevering 5}
\author{Jacob Oliver Bruun}
\date{\today}

\makeatletter
\let\inserttitle\@title
\let\insertauthor\@author
\makeatother

\pagestyle{fancy}
\chead{\insertauthor}
\lhead{\inserttitle}
\rhead{\today}

\parindent 0ex

\begin{document}

\section*{12.1.1}
Skal vise fundementalteoremet for PDEer av andre orden og lar dermed $u_{1}$ og $u_{2}$ oppfylle følgende likning
\begin{equation}
  \label{eq:4}
  au_{xx} + bu_{yy} + cu_{zz} + du_{xy} + fu_{xz} + gu_{yz} + hu_{x} + ku_{y} + mu_{z} + nu = p
\end{equation}
og $u = ku_{1} + cu_{2}, \;\; \forall (c,k) \in \M{R}{}$



\section*{12.1.14d}
Ser på


\section*{12.1.15}
Har at $u(x, y) = a \ln (x^{2} + y^{2}) + b$ er en løsning på Laplacelikningen, $\nabla^{2} u = 0$, og kan sjekke dette ved
\begin{equation}
  \label{eq:1}
  \begin{split}
    \nabla^{2} u &= \frac{\partial^{2}}{\partial x^{2}} a \ln \nbrack{ x^{2} + y^{2} } + \frac{\partial^{2}}{\partial y^{2}} a \ln \nbrack{ x^{2} + y^{2} } \\
                 &= \frac{\partial}{\partial x} \bbrack{ \frac{2ax}{x^{2} + y^{2}} } + \frac{\partial}{\partial y} \bbrack{ \frac{2ay}{x^{2} + y^{2}} } \\
                 &= \frac{ 2a\nbrack{ x^{2} + y^{2} } - 4ax^{2} }{ \nbrack{ x^{2} + y^{2} }^{2} } +  \frac{ 2a\nbrack{ x^{2} + y^{2} } - 4ay^{2} }{ \nbrack{ x^{2} + y^{2} }^{2} } \\
                 &= 0
  \end{split}
\end{equation}
og har dermed at $u$ oppfyller Laplacelikningen, skal så finne den spesielle løsningen for $u \nbrack{ x^{2} + y^{2} = 1 } = 110$ og $u \nbrack{ x^{2} + y^{2} = 100 } = 0$ og har da likningssettet
\begin{equation}
  \label{eq:2}
  \left\lbrace
  \begin{array}{rcl}
    a\ln(1) + b & = & 110 \\
    a\ln(100) + b & = & 0 \\
  \end{array} \right. \Rightarrow
  b = 110, \;\; a = - \frac{110}{\ln (100)}
\end{equation}
og har dermed
\begin{equation}
  \label{eq:3}
  u(x, y) = - \frac{110}{\ln (100)} \ln\nbrack{ x^{2} + y^{2} } + 110
\end{equation}



\section*{12.3.5}
Skal finne $u(x, t)$ for en streng med lengde $L=1$ og $c^{2} = 1$ med $u(x, 0) = k \sin 3\pi x$ med $k = 0.01$ og har da likningen
\begin{equation}
  \label{eq:5}
  \left\lbrace
    \begin{array}{ll}
      \partial_{xx} u = \partial_{tt} u &, x \in (0, 1) \\
      u(0, t) = 0 = u(L, t) &, t \geq 0\\
      u(x, 0) = k \sin 3\pi x &, x \in \bbrack{ 0, 1 }
    \end{array} \right.
\end{equation}
antar $u(x, t) = F(x) \cdot G(t)$ og har da
\begin{equation}
  \label{eq:6}
  F''(x) G(t) = G''(t) F(x) \Leftrightarrow \frac{F''(x)}{F(x)} = \frac{G''(t)}{G(t)} = \kappa
\end{equation}
for en konstant $\kappa$ og har da likningen
\begin{equation}
  \label{eq:7}
  \left\lbrace
    \begin{array}{l}
      F'' - \kappa F = 0 \\
      G'' - \kappa G = 0
    \end{array}
  \right.
\end{equation}
vi ser fra initialbetingelsene at $u(0, t) = F(0)G(t) = 0 = F(L)G(t) = u(L, t)$ gir $G(t) = 0 \lor F(0) = 0 = F(L)$ og siden $G(t) = 0$ ikke er en løsning vi er ute etter har vi dermed $F(0) = 0 = F(L)$ og ser da at $\kappa = 0$ gir $F(x) = ax + b$ og $F(0) = 0$ gir $b=0$ og $F(L) = 0$ gir $a=0$ som ikke er en relevant løsning. Ser på $\kappa = q^{2}$ for $q \in \M{R}{}$ og har da
\begin{equation}
  \label{eq:9}
  F(x) = Ae^{qx} + Be^{-qx}
\end{equation}
har så ved initialbetingelsene
\begin{equation}
  \label{eq:10}
  \left\lbrace
    \begin{array}{l}
      A + B = 0 \\
      Ae^{qL} + Be^{-qL} = 0
    \end{array}
  \right. \Rightarrow \left\lbrace
    \begin{array}{l}
      A = -B \\
      Ae^{qL} - Ae^{-qL} = 0
    \end{array}
  \right.
\end{equation}
og har dermed $A=B=0$ og kan så sette $\kappa = -q^{2}$ og får
\begin{equation}
  \label{eq:11}
  F(x) = C\sin qx + D\cos qx
\end{equation}
$F(0) = D = 0$ og $F(L) = C\sin qL = 0$ og siden $C = 0$ gir $F(x) = 0$ noe vi ikke er interesserte i setter vi $C \neq 0$ og finner da
\begin{equation}
  \label{eq:13}
  q = \frac{\pi n}{L}, \;\; n \in \M{N}{} \cup {0}
\end{equation}
siden $\sin -x = -\sin x$ og setter $C = 1$ og har da uendelig mange løsninger
\begin{equation}
  \label{eq:14}
  F_{n}(x) = \sin \frac{\pi n}{L} x
\end{equation}
fra \eqref{eq:7} har vi
\begin{equation}
  \label{eq:8}
  G'' + q^{2} G = 0
\end{equation}
og har dermed
\begin{equation}
  \label{eq:12}
  G_{n}(t) = \alpha_{n} \sin \frac{\pi n}{L} t + \beta_{n} \cos \frac{\pi n}{L} t
\end{equation}
og har dermed
\begin{equation}
  \label{eq:15}
  u_{n}(x, t) = \nbrack{ \alpha_{n} \sin \frac{\pi n}{L} t + \beta_{n} \cos \frac{\pi n}{L} t } \sin \frac{\pi n}{L} x
\end{equation}
ser $u_{0}(x, t) = 0$ og har den generelle løsningen
\begin{equation}
  \label{eq:16}
  u(x, t) = \sum_{n=1}^{\infty} u_{n} = \sum_{n=1}^{\infty} \nbrack{ \alpha_{n} \sin \frac{\pi n}{L} t + \beta_{n} \cos \frac{\pi n}{L} t } \sin \frac{\pi n}{L} x
\end{equation}
har $u(x, 0) = k\sin 3\pi x$ og har dermed
\begin{equation}
  \label{eq:17}
  k\sin 3\pi x = \sum_{n=1}^{\infty} \beta_{n} \sin \frac{\pi n}{L}x
\end{equation}
ser at dette vil være en sinus-fourierrekke for $k\sin 3\pi x$ og har da med $L = 1$
\begin{equation}
  \label{eq:18}
  \begin{split}
    \beta_{n} &= 2 \int_{0}^{1} k\sin 3\pi x \sin \pi nx dx \\
              &= 2k \bbrack{ -\frac{1}{3\pi} \cos 3\pi x \sin \pi nx \Big|_{0}^{1} + \frac{n}{3} \int_{0}^{1} \cos 3\pi x \cos \pi nx dx } \\
  \end{split}
\end{equation}


\section*{12.3.7}
Bruker samme utledning som i forrige oppgave og har
\begin{equation}
  \label{eq:19}
  \begin{split}
    \beta_{n} &= 2 \int_{0}^{1} kx(1-x) \sin \pi nx dx \\
              &= 2k \int \bbrack{ x\sin \pi nx - x^{2}\sin \pi nx } dx \\
    &= 2k \bbrack{ \frac{(-1)^{n+1}}{\pi n} - \frac{ \nbrack{ 2 - \pi ^{2} n^{2} } (-1)^{n} - 2 }{\pi^{3} n^{3}} }
  \end{split}
\end{equation}







\end{document}
