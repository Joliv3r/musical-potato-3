\documentclass{report}

\usepackage{amsfonts, amsmath, amssymb, amsthm}
\usepackage[margin=1.0in]{geometry}
\usepackage{hyperref}
\usepackage{float}
\usepackage{fancyhdr}
\usepackage{graphicx}
\usepackage{mathrsfs}
\usepackage{comment}

\newcommand{\M}[2]{\mathbb{#1}^{#2}}
\newcommand{\twovector}[2]{\left[ \begin{array}{c} #1 \\ #2 \\ \end{array} \right]}
\newcommand{\threevector}[3]{\left[ \begin{array}{c} #1 \\ #2 \\ #3 \\ \end{array} \right]}
\newcommand{\fourvector}[4]{\left[ \begin{array}{c} #1 \\ #2 \\ #3 \\ #4 \\ \end{array} \right]}
\newcommand{\fivevector}[5]{\left[ \begin{array}{c} #1 \\ #2 \\ #3 \\ #4 \\ #5 \\ \end{array} \right]}
\newcommand{\sixvector}[6]{\left[ \begin{array}{c} #1 \\ #2 \\ #3 \\ #4 \\ #5 \\ #6 \\ \end{array} \right]}
\newcommand{\sevenvector}[7]{\left[ \begin{array}{c} #1 \\ #2 \\ #3 \\ #4 \\ #5 \\ #6 \\ #7 \\ \end{array} \right]}
\newcommand{\eightvector}[8]{\left[ \begin{array}{c} #1 \\ #2 \\ #3 \\ #4 \\ #5 \\ #6 \\ #7 \\ #8 \\ \end{array} \right]}
\newcommand{\ninevector}[9]{\left[ \begin{array}{c} #1 \\ #2 \\ #3 \\ #4 \\ #5 \\ #6 \\ #7 \\ #8 \\ #9 \\ \end{array} \right]}
\newcommand{\nbrack}[1]{\left( #1 \right)}
\newcommand{\bbrack}[1]{\left[ #1 \right]}
\newcommand{\cbrack}[1]{\left\lbrace #1 \right\rbrace}
\newcommand{\abrack}[1]{\left< #1 \right>}
\newcommand{\linebrack}[1]{\left| #1 \right|}
\newcommand{\twomatrix}[4]{\bbrack{
    \begin{array}{cc}
      #1 & #2 \\
      #3 & #4 \\
    \end{array}
  }
}
\newcommand{\threematrix}[9]{\bbrack{
    \begin{array}{ccc}
      #1 & #2 & #3 \\
      #4 & #5 & #6 \\
      #7 & #8 & #9 \\
    \end{array}
  }
}
\newcommand{\Lplc}[1]{\mathscr{L}\bbrack{ #1 } (s)}
\newcommand{\iLplc}[1]{\mathscr{L}^{-1}\bbrack{ #1 } (t)}
\newcommand{\til}[1]{\overset{~}{#1}}

\title{Innlevering 5}
\author{Jacob Oliver Bruun}
\date{\today}

\makeatletter
\let\inserttitle\@title
\let\insertauthor\@author
\makeatother

\pagestyle{fancy}
\chead{\insertauthor}
\lhead{\inserttitle}
\rhead{\today}

\parindent 0ex

\begin{document}

\section*{12.1.14d}
Ser på


\section*{12.1.15}
Har at $u(x, y) = a \ln (x^{2} + y^{2}) + b$ er en løsning på Laplacelikningen, $\nabla^{2} u = 0$, og kan sjekke dette ved
\begin{equation}
  \label{eq:1}
  \begin{split}
    \nabla^{2} u &= \frac{\partial^{2}}{\partial x^{2}} a \ln \nbrack{ x^{2} + y^{2} } + \frac{\partial^{2}}{\partial y^{2}} a \ln \nbrack{ x^{2} + y^{2} } \\
                 &= \frac{\partial}{\partial x} \bbrack{ \frac{2ax}{x^{2} + y^{2}} } + \frac{\partial}{\partial y} \bbrack{ \frac{2ay}{x^{2} + y^{2}} } \\
                 &= \frac{ 2a\nbrack{ x^{2} + y^{2} } - 4ax^{2} }{ \nbrack{ x^{2} + y^{2} }^{2} } +  \frac{ 2a\nbrack{ x^{2} + y^{2} } - 4ay^{2} }{ \nbrack{ x^{2} + y^{2} }^{2} } \\
                 &= 0
  \end{split}
\end{equation}
og har dermed at $u$ oppfyller Laplacelikningen, skal så finne den spesielle løsningen for $u \nbrack{ x^{2} + y^{2} = 1 } = 110$ og $u \nbrack{ x^{2} + y^{2} = 100 } = 0$ og har da likningssettet
\begin{equation}
  \label{eq:2}
  \left\lbrace
  \begin{array}{rcl}
    a\ln(1) + b & = & 110 \\
    a\ln(100) + b & = & 0 \\
  \end{array} \right. \Rightarrow
  b = 110, \;\; a = - \frac{110}{\ln (100)}
\end{equation}
og har dermed
\begin{equation}
  \label{eq:3}
  u(x, y) = - \frac{110}{\ln (100)} \ln\nbrack{ x^{2} + y^{2} } + 110
\end{equation}



\end{document}
