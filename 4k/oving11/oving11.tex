\documentclass{report}

\usepackage{amsfonts, amsmath, amssymb, amsthm}
\usepackage[margin=1.0in]{geometry}
\usepackage{hyperref}
\usepackage{float}
\usepackage{fancyhdr}
\usepackage{graphicx}
\usepackage{mathrsfs}
\usepackage{comment}
\usepackage{mathtools}

\newcommand{\M}[2]{\mathbb{#1}^{#2}}
\newcommand{\twovector}[2]{\left[ \begin{array}{c} #1 \\ #2 \\ \end{array} \right]}
\newcommand{\threevector}[3]{\left[ \begin{array}{c} #1 \\ #2 \\ #3 \\ \end{array} \right]}
\newcommand{\fourvector}[4]{\left[ \begin{array}{c} #1 \\ #2 \\ #3 \\ #4 \\ \end{array} \right]}
\newcommand{\fivevector}[5]{\left[ \begin{array}{c} #1 \\ #2 \\ #3 \\ #4 \\ #5 \\ \end{array} \right]}
\newcommand{\sixvector}[6]{\left[ \begin{array}{c} #1 \\ #2 \\ #3 \\ #4 \\ #5 \\ #6 \\ \end{array} \right]}
\newcommand{\sevenvector}[7]{\left[ \begin{array}{c} #1 \\ #2 \\ #3 \\ #4 \\ #5 \\ #6 \\ #7 \\ \end{array} \right]}
\newcommand{\eightvector}[8]{\left[ \begin{array}{c} #1 \\ #2 \\ #3 \\ #4 \\ #5 \\ #6 \\ #7 \\ #8 \\ \end{array} \right]}
\newcommand{\ninevector}[9]{\left[ \begin{array}{c} #1 \\ #2 \\ #3 \\ #4 \\ #5 \\ #6 \\ #7 \\ #8 \\ #9 \\ \end{array} \right]}
\newcommand{\nbrack}[1]{\left( #1 \right)}
\newcommand{\bbrack}[1]{\left[ #1 \right]}
\newcommand{\cbrack}[1]{\left\lbrace #1 \right\rbrace}
\newcommand{\abrack}[1]{\left< #1 \right>}
\newcommand{\linebrack}[1]{\left| #1 \right|}
\newcommand{\twomatrix}[4]{\bbrack{
    \begin{array}{cc}
      #1 & #2 \\
      #3 & #4 \\
    \end{array}
  }
}
\newcommand{\threematrix}[9]{\bbrack{
    \begin{array}{ccc}
      #1 & #2 & #3 \\
      #4 & #5 & #6 \\
      #7 & #8 & #9 \\
    \end{array}
  }
}
\newcommand{\Lplc}[1]{\mathscr{L}\bbrack{ #1 } (s)}
\newcommand{\iLplc}[1]{\mathscr{L}^{-1}\bbrack{ #1 } (t)}
\newcommand{\im}{\text{Im}}
\newcommand{\re}{\text{Re}}
\newcommand{\limtoinf}[1]{\lim_{#1 \rightarrow \infty}}
\newcommand{\Res}[1]{\underset{#1}{\text{Res}}}

\title{Innlevering 10}
\author{Jacob Oliver Bruun}
\date{\today}

\makeatletter
\let\inserttitle\@title
\let\insertauthor\@author
\makeatother

\pagestyle{fancy}
\chead{\insertauthor}
\lhead{\inserttitle}
\rhead{\today}

\parindent 0ex

\begin{document}
\section*{15.3.5}
Skal finne konvergensradius for rekken
\begin{equation}
  \label{eq:1}
  f(z) = \sum_{n=2}^{\infty} \frac{n(n-1)}{4^{n}} \nbrack{ z-2i }^{n}
\end{equation}
bruker Cauchy-Hardamard og finner konvergensradiusen
\begin{equation}
  \label{eq:2}
  R = \limtoinf{n} \frac{ n(n-1) \cdot 4^{n+1} }{4^{n} \cdot (n+1)n} = \limtoinf{n} \frac{4n - 4}{n + 1} = 4
\end{equation}
kan også finne konvergensradiusen ved å derivere rekken to ganger
\begin{equation}
  \label{eq:3}
  f''(z) = \sum_{n = 2}^{\infty} 4^{-n} \nbrack{z-2i}^{n-2} = \sum_{n=0}^{\infty} 4^{-n+2} \nbrack{z-2i}^{n} = 4^{2} \sum_{n=0}^{\infty} \nbrack{ \frac{z}{4} - \frac{i}{2} }^{n}
\end{equation}
dette er en geometrisk rekke og vet at den konvergerer for
\begin{equation}
  \label{eq:4}
  \linebrack{ \frac{z}{4} - \frac{i}{2} } < 1 \Rightarrow \linebrack{ z - 2i } < 4
\end{equation}
og har dermed også her konvergensradius $R = 4$.


\section*{15.3.8}
Ser på rekken
\begin{equation}
  \label{eq:5}
  f(z) = \sum_{n=1}^{\infty} \frac{3^{n}}{n(n+1)}z^{n}
\end{equation}
og finner konvergensradius ved Cauchy-Hardamard
\begin{equation}
  \label{eq:6}
  R = \limtoinf{n} \frac{3^{n} \cdot (n+1)(n+2)}{n(n+1) \cdot 3^{n+1}} = \frac{1}{3} \limtoinf{n} \frac{n+2}{n} = \frac{1}{3}
\end{equation}
kan også integrere funksjonen to ganger og får da
\begin{equation}
  \label{eq:7}
  \iint f(z) dz dz = \
\end{equation}


\section*{15.3.10}
Ser på rekken
\begin{equation}
  \label{eq:8}
  f(z) = \sum_{n=k}^{\infty} \binom{n}{k} \nbrack{\frac{z}{2}}^{n} = \sum_{n=k}^{\infty} \frac{n!}{2^{n} \cdot k! \nbrack{ n-k }!} z^{n}
\end{equation}
og finner ved Cauchy-Hardamard
\begin{equation}
  \label{eq:9}
  R = \limtoinf{n} \frac{n! \cdot 2^{n+1} k! (n+1-k)!}{2^{n}k! (n-k)! \cdot (n+1)!} = \limtoinf{n} \frac{2 (n-k+1)}{(n+1)} = 2
\end{equation}


\section*{15.4.3}
Ser på
\begin{equation}
  \label{eq:10}
  f(z) = \sin \frac{z^{2}}{2}
\end{equation}
og vil finne Maclaurin-rekken som representerer $f(z)$ og setter $u = z^{2}/2$ og kan da skrive
\begin{equation}
  \label{eq:14}
  \sin \frac{z^{2}}{2} = \sin u = \sum_{n=0}^{\infty} \frac{(-1)^{n} u^{1+2n}}{(1+2n)!} = \sum_{n=0}^{\infty} \frac{(-1)^{n} z^{2+4n}}{2^{1+2n}(1+2n)!} = \frac{z^{2}}{2} \sum_{n=0}^{\infty} \frac{(-1)^{n} z^{4n}}{2^{2n}(1+2n)!}
\end{equation}
for å finne konvergens setter vi $u = z^{4}$ og ser kun på selve rekka
\begin{equation}
  \label{eq:16}
  \sum_{n=0}^{\infty} \frac{(-1)^{n}u^{n}}{2^{2n}(1+2n)!}
\end{equation}
kan bruke Cauchy-Hadamard og finner konvergensradiusen
\begin{equation}
  \label{eq:17}
  R = \limtoinf{n} \linebrack{ \frac{ 2^{2n + 2} (2n + 3)! }{ 2^{2n}(1+2n)! } } \xrightarrow{n \to \infty} \infty
\end{equation}
og Maclaurin-rekken konvergerer for alle $z\in \M{C}{}$.


\section*{15.4.4}
Ser så på
\begin{equation}
  \label{eq:18}
  f(z) = \frac{z+2}{1-z^{2}} = \frac{z+2}{(1-z)(1+z)}
\end{equation}
og kan finne
\begin{equation}
  \label{eq:19}
  a_{n} = \frac{1}{2\pi i} \oint_{\mathcal{C}} \frac{f(z^{*})}{(z^{*} - z_{0})^{n+1}}dz^{*}
\end{equation}
og setter $z_{0} = 0$ og definerer
\begin{equation}
  \label{eq:20}
  g(z) = \frac{f(z)}{z^{n+1}}, \;\;\; n \in \M{N}{}
\end{equation}
og har dermed en pol av orden $n+1$ i $z = 0$ og to poler av orden 1 i $z=\pm 1$ og kan velge $\mathcal{C}$ slik at den inneholder $z = z_{0} = 0$, men ikke $z = \pm 1$ og kan finne
\begin{equation}
  \label{eq:21}
  \Res{z=0} f(z) =
\end{equation}




\section*{15.4.8}
\section*{15.4.23}
\section*{15.4.24}
\section*{16.1.2}
\section*{16.1.6}
\section*{16.1.13}




\end{document}
