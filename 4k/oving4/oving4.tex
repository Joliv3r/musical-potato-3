\documentclass{report}

\usepackage{amsfonts, amsmath, amssymb, amsthm}
\usepackage[margin=1.0in]{geometry}
\usepackage{hyperref}
\usepackage{float}
\usepackage{fancyhdr}
\usepackage{graphicx}
\usepackage{mathrsfs}
\usepackage{comment}

\newcommand{\M}[2]{\mathbb{#1}^{#2}}
\newcommand{\twovector}[2]{\left[ \begin{array}{c} #1 \\ #2 \\ \end{array} \right]}
\newcommand{\threevector}[3]{\left[ \begin{array}{c} #1 \\ #2 \\ #3 \\ \end{array} \right]}
\newcommand{\fourvector}[4]{\left[ \begin{array}{c} #1 \\ #2 \\ #3 \\ #4 \\ \end{array} \right]}
\newcommand{\fivevector}[5]{\left[ \begin{array}{c} #1 \\ #2 \\ #3 \\ #4 \\ #5 \\ \end{array} \right]}
\newcommand{\sixvector}[6]{\left[ \begin{array}{c} #1 \\ #2 \\ #3 \\ #4 \\ #5 \\ #6 \\ \end{array} \right]}
\newcommand{\sevenvector}[7]{\left[ \begin{array}{c} #1 \\ #2 \\ #3 \\ #4 \\ #5 \\ #6 \\ #7 \\ \end{array} \right]}
\newcommand{\eightvector}[8]{\left[ \begin{array}{c} #1 \\ #2 \\ #3 \\ #4 \\ #5 \\ #6 \\ #7 \\ #8 \\ \end{array} \right]}
\newcommand{\ninevector}[9]{\left[ \begin{array}{c} #1 \\ #2 \\ #3 \\ #4 \\ #5 \\ #6 \\ #7 \\ #8 \\ #9 \\ \end{array} \right]}
\newcommand{\nbrack}[1]{\left( #1 \right)}
\newcommand{\bbrack}[1]{\left[ #1 \right]}
\newcommand{\cbrack}[1]{\left\lbrace #1 \right\rbrace}
\newcommand{\abrack}[1]{\left< #1 \right>}
\newcommand{\linebrack}[1]{\left| #1 \right|}
\newcommand{\twomatrix}[4]{\bbrack{
    \begin{array}{cc}
      #1 & #2 \\
      #3 & #4 \\
    \end{array}
  }
}
\newcommand{\threematrix}[9]{\bbrack{
    \begin{array}{ccc}
      #1 & #2 & #3 \\
      #4 & #5 & #6 \\
      #7 & #8 & #9 \\
    \end{array}
  }
}
\newcommand{\Lplc}[1]{\mathscr{L}\bbrack{ #1 } (s)}
\newcommand{\iLplc}[1]{\mathscr{L}^{-1}\bbrack{ #1 } (t)}
\newcommand{\fft}[1]{\mathcal{F} \bbrack{ #1 } (w)}

\title{Innlevering 4}
\author{Jacob Oliver Bruun}
\date{\today}

\makeatletter
\let\inserttitle\@title
\let\insertauthor\@author
\makeatother

\pagestyle{fancy}
\chead{\insertauthor}
\lhead{\inserttitle}
\rhead{\today}

\parindent 0ex

\begin{document}
\section*{11.4.2}
Ser på Fourier-konstantene til $f(x) = x$ og ser at funksjonen er odde og har dermed $a_{n} = 0, n \in \M{N}{}$ og ser så på
\begin{equation}
  \label{eq:2}
  b_{n} = \frac{2}{\pi} \int_{0}^{\pi} x \sin nx dx
  = \frac{2}{\pi} \bbrack{ -\frac{x}{n} \cos nx \Big|_{0}^{\pi} + \frac{1}{n} \int_{0}^{\pi} \cos nx dx } = \frac{2}{n} (-1)^{n+1}
\end{equation}
ser så på kvadrat feilen
\begin{equation}
  \label{eq:4}
  E_{k} = || f(x) - S_{f, k}(x) ||^{2} = \int_{-\pi}^{\pi} \nbrack{ f(x) }^{2} dx - \pi \bbrack{ 2a_{0}^{2} + \sum_{n=1}^{k} \nbrack{ a_{n}^{2} + b_{n}^{2} }} = \int_{-\pi}^{\pi} x^{2}dx - \pi \sum_{n=1}^{k}b_{n}^{2}
\end{equation}
og har for alle disse utregningene
\begin{equation}
  \label{eq:6}
  \int_{-\pi}^{\pi} \nbrack{ f(x) }^{2}dx = \int_{-\pi}^{\pi} x^{2} dx = \frac{2\pi^{3}}{3}
\end{equation}
har dermed
\begin{equation}
  \label{eq:7}
  E_{k} = \frac{2\pi^{3}}{3} - \pi \sum_{n=1}^{k} \frac{4}{n^{2}}
\end{equation}
og finner så for ulike verdier av $k$
\begin{equation}
  \label{eq:8}
  \begin{split}
    E_{1} &= 8.104 \\
    E_{2} &= 4.963 \\
    E_{3} &= 3.567 \\
    E_{4} &= 2.781 \\
    E_{5} &= 2.279 \\
  \end{split}
\end{equation}


\section*{11.4.3}
Har nå $f(x) = |x|$ for $-\pi < x < \pi$ og ser at dette kun er en like periodisk utvidning av $x$ for $0 < x < \pi$ og kan dermed finne Fourier-koeffisientene med $b_{n} = 0$
\begin{equation}
  \label{eq:9}
  a_{0} = \frac{1}{\pi} \int_{0}^{\pi} f(x) dx = \frac{\pi}{2}
\end{equation}
og
\begin{equation}
  \label{eq:10}
  a_{n} = \frac{2}{\pi} \int_{0}^{\pi} x \cos nx dx = \frac{2}{n\pi} \bbrack{ x\sin nx \Big|_{0}^{\pi} - \int_{0}^{\pi} \sin nx dx } = \frac{2}{n^{2} \pi} \nbrack{ \cos n\pi - 1 } = \frac{4}{n^{2}\pi}
\end{equation}
for $n = 2k-1, k\in\M{N}{}$ og har at $x^{2} = |x|^{2}$ og kan bruke tidligere utregninger og får da
\begin{equation}
  \label{eq:12}
  E_{k} = \frac{2\pi^{3}}{3} -  \pi \bbrack{ \frac{\pi^{2}}{2} + \sum_{n=1}^{k} \frac{16}{n^{4}\pi^{2}}}
  = \frac{\pi^{3}}{6} - \pi \sum_{n=1}^{k} \frac{16}{n^{4}\pi^{2}}
\end{equation}
og har da
\begin{equation}
  \label{eq:13}
  \begin{split}
    E_{1} &= 0.075\\
    E_{2} &= 0.075 \\
    E_{3} &= 0.0119\\
    E_{4} &= 0.0119\\
    E_{5} &= 0.00373\\
  \end{split}
\end{equation}




\section*{11.4.13}
\begin{proof}
  Ser på rekken vi har fått vite konvergensen til
  \begin{equation}
    \label{eq:14}
    1 + \frac{1}{3^{4}} + \frac{1}{5^{4}} + \frac{1}{7^{4}} + \dots = \sum_{n=1}^{\infty} \frac{1}{(2n-1)^{4}} = \frac{\pi^{4}}{96}
  \end{equation}
  har Parsevals identitet
  \begin{equation}
    \label{eq:15}
    2a_{0}^{2} + \sum_{n=1}^{\infty} \nbrack{ a_{n}^{2} + b_{n}^{2} } = \frac{1}{\pi} \int_{-\pi}^{\pi} f(x)^{2} dx
  \end{equation}
  kan anta en odde funksjon slik at $b_{n} = 0 \forall n\in\M{N}{}$ og kan da finne en funksjon slik at
  \begin{equation}
    \label{eq:16}
    a_{n} = \frac{1}{2n^{2}} \bbrack{ (-1)^{n+1} + 1 } = \frac{1}{2n^{2}}\bbrack{ 1 - \cos n\pi }
  \end{equation}
  ser at dersom en bruker delvis integrasjon får vi $n^{2}$ og kan tenke på dette som et polynom $f(x) = ax$
  \begin{equation}
    \label{eq:5}
    a_{n} = \frac{2a}{\pi} \int_{0}^{\pi} x\cos nx dx
  \end{equation}
  og fra tidligere oppgave har vi integralet og dermed
  \begin{equation}
    \label{eq:11}
    a_{n} = \frac{2a}{n^{2}\pi}\nbrack{ \cos n\pi - 1 }
  \end{equation}
  setter vi $a = \pi /4$ har vi
  \begin{equation}
    \label{eq:17}
    a_{0} = \frac{1}{4} \int_{0}^{\pi} x dx = \frac{\pi^{2}}{8}
  \end{equation}
  og har også
  \begin{equation}
    \label{eq:18}
    \frac{1}{\pi} \int_{-\pi}^{\pi} f(x)^{2} dx = \frac{\pi}{8} \int_{0}^{\pi} x^{2} dx = \frac{\pi^{4}}{24}
  \end{equation}
  og ser da videre fra Parsevals identitet
  \begin{equation}
    \label{eq:19}
    \begin{split}
      2\cdot \frac{\pi^{4}}{64} + \sum_{n=1}^{\infty} \frac{1}{\nbrack{2n-1}^{2}} &= \frac{\pi^{4}}{24} \\
      \sum_{n=1}^{\infty} \frac{1}{\nbrack{2n-1}^{2}} &= \frac{\pi^{4}}{24} - \frac{2 \pi^{4}}{64} \\
      \sum_{n=1}^{\infty} \frac{1}{\nbrack{2n-1}^{2}} &= \frac{\pi^{4}}{96}
    \end{split}
  \end{equation}
\end{proof}


\section*{11.4.9}
Har $f(x) = x$ for $-\pi < x < \pi$ og har da den komplekse Fourierrekka
\begin{equation}
  \label{eq:20}
  f(x) \sim \sum_{n=-\infty}^{\infty} c_{n} e^{\frac{in\pi x}{L}}
\end{equation}
og kan regne ut
\begin{equation}
  \label{eq:21}
  \begin{split}
    c_{n} &= \frac{1}{2\pi} \int_{-\pi}^{\pi} xe^{inx} dx \\
          &= \frac{1}{2\pi in} \bbrack{ xe^{inx}\Big|_{-\pi}^{\pi} - \int_{-\pi}^{\pi} e^{inx} dx } \\
          &= \frac{1}{2\pi in} \bbrack{ \pi \nbrack{ e^{in\pi} + e^{-in\pi} } - \frac{1}{in} \nbrack{ e^{in\pi} - e^{-in\pi} } } \\
          &= \frac{1}{2in}\nbrack{ e^{in\pi} + e^{-in\pi} } + \frac{1}{2\pi n^{2}} \nbrack{ e^{in\pi} - e^{-in\pi} } \\
          &= \frac{1}{in} \cos n\pi + \frac{i}{\pi n^{2}} \sin n\pi
  \end{split}
\end{equation}
ser så hele rekka
\begin{equation}
  \label{eq:22}
  \begin{split}
    f(x) &= \sum_{-\infty}^{\infty} \cbrack{ \frac{1}{2in} \bbrack{ e^{in(\pi + x)} + e^{in(x - \pi)} } + \frac{1}{2\pi n^{2}} \bbrack{ e^{in(\pi + x)} - e^{in(x-\pi)} } }
  \end{split}
\end{equation}



\section*{11.7.1}
\begin{proof}
  Ser på
  \begin{equation}
    \label{eq:3}
    f(x) = \int_{0}^{\infty} \frac{\cos xw + w \sin xw}{1 + w^{2}}dw = \left\lbrace
    \begin{array}{ll}
      0 & , x < 0 \\
      \frac{\pi}{2} & , x = 0 \\
      \pi e^{-x} & , x > 0 \\
    \end{array} \right.
  \end{equation}
  ser at vi har formen
  \begin{equation}
    \label{eq:23}
    f(x) = \int_{0}^{\infty} \bbrack{ A(w)\cos wx + B(w) \sin wx } dw, \;\; A(w) = \frac{1}{1+w^{2}}, \; B(w) = \frac{w}{1+w^{2}}
  \end{equation}
  og har at
  \begin{equation}
    \label{eq:24}
    A(w) = \frac{1}{\pi} \int_{-\infty}^{\infty} f(v) \cos wv dv, \;\; B(w) = \frac{1}{\pi} \int_{-\infty}^{\infty} f(v) \sin wv dv
  \end{equation}
  ser at funksjonen er 0 for $x < 0$ og har dermed
  \begin{equation}
    \label{eq:25}
    A(w) = \frac{1}{\pi} \int_{0}^{\infty} \pi e^{-v} \cos wv dv, \;\; B(w) = \frac{1}{\pi} \int_{0}^{\infty} \pi e^{-v} \sin wv dv
  \end{equation}
  og kan fra dette finne $A(w)$ og $B(w)$ og viss de er de samme som gitt i \eqref{eq:23} så holder funksjonen
  \begin{equation}
    \label{eq:1}
    \begin{split}
      A(w) &= \int_{0}^{\infty} \frac{e^{-v}}{2} \bbrack{ e^{iwv} + e^{-iwv} } dv \\
          &= \frac{1}{2} \int_{0}^{\infty} \bbrack{ e^{v(iw-1)} + e^{-v(iw+1)} } dv \\
          &= \frac{1}{2} \bbrack{ \frac{e^{v(iw-1)}}{iw-1} - \frac{e^{-v(iw+1)}}{iw+1} }_{0}^{\infty} \\
          &= \frac{1}{2} \bbrack{ \frac{e^{viw}e^{-v}}{iw-1} - \frac{e^{-viw}e^{-v}}{iw+1} }_{0}^{\infty} \\
          &= \frac{1}{2} \bbrack{ -\frac{1}{iw - 1} + \frac{1}{iw+1} } \\
          &= \frac{1}{2} \bbrack{ \frac{iw-1 - iw-1}{-w^{2}-1} } \\
          &= \frac{1}{w^{2} + 1}
    \end{split}
  \end{equation}
  ser så på $B(w)$
  \begin{equation}
    \label{eq:26}
    \begin{split}
      B(w) &= \int_{0}^{\infty} \frac{e^{-v}}{2i} \bbrack{ e^{iwv} - e^{-iwv} }dv \\
          &= \frac{1}{2i} \bbrack{ \frac{e^{iwv}e^{-v}}{iw-1} + \frac{e^{-iwv}e^{-v}}{iw + 1} }_{0}^{\infty} \\
          &= \frac{1}{2i} \bbrack{ -\frac{1}{iw-1} - \frac{1}{iw+1} } \\
          &= \frac{i}{2} \cdot \frac{iw+1 + iw-1}{-w^{2} - 1} \\
          &= \frac{w}{w^{2} + 1}
    \end{split}
  \end{equation}
\end{proof}



\section*{11.9.5}
Skal finne Fouriertransformen til
\begin{equation}
  \label{eq:27}
  f(x) = \left\lbrace
    \begin{array}{ll}
      e^{x} & , -a < x < a \\
      0 & , \text{ellers}
    \end{array} \right.
\end{equation}
har da
\begin{equation}
  \label{eq:28}
  \begin{split}
    \fft{f} &= \frac{1}{\sqrt{2\pi}} \int_{-\infty}^{\infty} f(x) e^{-iwx} dx \\
            &= \frac{1}{\sqrt{2\pi}} \int_{-a}^{a} e^{x(1-iw)} dx \\
            &= \frac{1}{\sqrt{2\pi}} \bbrack{\frac{e^{x}e^{-iwx}}{1-iw}}_{-a}^{a} \\
            &= \frac{e^{a}e^{-aiw} - e^{-a}e^{aiw}}{\sqrt{2\pi} \nbrack{ 1-iw }} \\
  \end{split}
\end{equation}



\section*{11.9.7}
Skal finne Fouriertransformen til
\begin{equation}
  \label{eq:29}
  f(x) = \left\lbrace
    \begin{array}{ll}
      x & , 0 < x < a \\
      0 & , \text{ellers}
    \end{array} \right.
\end{equation}
har da
\begin{equation}
  \label{eq:30}
  \begin{split}
    \fft{f} &= \frac{1}{\sqrt{2\pi}} \int_{0}^{a} xe^{-iwx} dx \\
            &= -\frac{1}{\sqrt{2\pi} iw} \bbrack{ xe^{-iwx} \Big|_{0}^{a} + \int_{0}^{a} e^{-iwx} dx } \\
            &= \frac{1}{\sqrt{2\pi} iw} \bbrack{ ae^{-iwx} - \frac{e^{-iwa}-1}{iw} } \\
            &= \frac{ae^{-iwx}}{\sqrt{2\pi}iw} + \frac{e^{-iwa} - 1}{\sqrt{2\pi}w^{2}} \\
  \end{split}
\end{equation}



\section*{11.9.9}
Skal finne Fouriertransformen til
\begin{equation}
  \label{eq:31}
  f(x) = \left\lbrace
    \begin{array}{ll}
      |x| & , -1 < x < 1 \\
      0 & , \text{ellers}
    \end{array} \right.
\end{equation}
har da
\begin{equation}
  \label{eq:32}
  \begin{split}
    \fft{f} &= \frac{1}{\sqrt{2\pi}} \int_{-1}^{1} |x|e^{-iwx} dx \\
            &= \frac{1}{\sqrt{2\pi}} \bbrack{ \int_{-1}^{0} -xe^{-iwx} dx + \int_{0}^{1} xe^{-iwx} dx } \\
            &= \frac{1}{\sqrt{2\pi} iw} \bbrack{ xe^{-iwx} \Big|_{-1}^{0} - \int_{-1}^{0} e^{-iwx} dx - xe^{-iwx} \Big|_{0}^{1} + \int_{0}^{1} e^{-iwx}dx } \\
            &= \frac{1}{\sqrt{2\pi}iw} \bbrack{ e^{iw} + \frac{1}{iw} - \frac{e^{iw}}{iw} - e^{-iw} - \frac{e^{-iw}}{iw} + \frac{1}{iw} } \\
            &= \frac{1}{\sqrt{2\pi} iw} \bbrack{ e^{iw} - e^{-iw} - \frac{e^{iw} + e^{-iw}}{iw} + \frac{2}{iw} } \\
            &= \frac{1}{\sqrt{2\pi}w} \bbrack{ 2\sin w + \frac{2}{w} \cos w - \frac{2}{w} } \\
            &= \frac{2w\sin w + 2\cos w - 2}{\sqrt{2\pi} w^{2}}
  \end{split}
\end{equation}










\end{document}
