\documentclass{report}

\usepackage{amsfonts, amsmath, amssymb, amsthm}
\usepackage[margin=1.0in]{geometry}
\usepackage{hyperref}
\usepackage{float}
\usepackage{fancyhdr}
\usepackage{graphicx}
\usepackage{mathrsfs}
\usepackage{comment}

\newcommand{\M}[2]{\mathbb{#1}^{#2}}
\newcommand{\twovector}[2]{\left[ \begin{array}{c} #1 \\ #2 \\ \end{array} \right]}
\newcommand{\threevector}[3]{\left[ \begin{array}{c} #1 \\ #2 \\ #3 \\ \end{array} \right]}
\newcommand{\fourvector}[4]{\left[ \begin{array}{c} #1 \\ #2 \\ #3 \\ #4 \\ \end{array} \right]}
\newcommand{\fivevector}[5]{\left[ \begin{array}{c} #1 \\ #2 \\ #3 \\ #4 \\ #5 \\ \end{array} \right]}
\newcommand{\sixvector}[6]{\left[ \begin{array}{c} #1 \\ #2 \\ #3 \\ #4 \\ #5 \\ #6 \\ \end{array} \right]}
\newcommand{\sevenvector}[7]{\left[ \begin{array}{c} #1 \\ #2 \\ #3 \\ #4 \\ #5 \\ #6 \\ #7 \\ \end{array} \right]}
\newcommand{\eightvector}[8]{\left[ \begin{array}{c} #1 \\ #2 \\ #3 \\ #4 \\ #5 \\ #6 \\ #7 \\ #8 \\ \end{array} \right]}
\newcommand{\ninevector}[9]{\left[ \begin{array}{c} #1 \\ #2 \\ #3 \\ #4 \\ #5 \\ #6 \\ #7 \\ #8 \\ #9 \\ \end{array} \right]}
\newcommand{\nbrack}[1]{\left( #1 \right)}
\newcommand{\bbrack}[1]{\left[ #1 \right]}
\newcommand{\cbrack}[1]{\left\lbrace #1 \right\rbrace}
\newcommand{\abrack}[1]{\left< #1 \right>}
\newcommand{\linebrack}[1]{\left| #1 \right|}
\newcommand{\twomatrix}[4]{\bbrack{
    \begin{array}{cc}
      #1 & #2 \\
      #3 & #4 \\
    \end{array}
  }
}
\newcommand{\threematrix}[9]{\bbrack{
    \begin{array}{ccc}
      #1 & #2 & #3 \\
      #4 & #5 & #6 \\
      #7 & #8 & #9 \\
    \end{array}
  }
}
\newcommand{\Lplc}[1]{\mathscr{L}\bbrack{ #1 } (s)}
\newcommand{\iLplc}[1]{\mathscr{L}^{-1}\bbrack{ #1 } (t)}
\newcommand{\til}[1]{\overset{~}{#1}}

\title{Innlevering 3}
\author{Jacob Oliver Bruun}
\date{\today}

\makeatletter
\let\inserttitle\@title
\let\insertauthor\@author
\makeatother

\pagestyle{fancy}
\chead{\insertauthor}
\lhead{\inserttitle}
\rhead{\today}

\parindent 0ex

\begin{document}
\section*{11.1.2}
Vet at fundamentalperioden for $\cos x$ er $2\pi$ og har dermed at fundamentalperioden til $\cos nx$ må være $p_{f} = 2\pi /n$ og har samme argumentet for $\sin nx$ og kan dermed konkludere med følgende
\begin{equation}
  \label{eq:1}
  \begin{split}
    \cos nx, \sin nx &\Rightarrow p_{f} = \frac{2\pi}{n} \\
    \cos \frac{2\pi}{k} x, \sin \frac{2\pi}{k} x &\Rightarrow p_{f} = k \\
    \cos \frac{2\pi n}{k} x, \sin \frac{2\pi n}{k} x &\Rightarrow p_{f} = \frac{k}{n} \\
  \end{split}
\end{equation}


\section*{11.1.15}
Ser på $f(x) = x^{2}$ for $0 < x < 2\pi$ med $f(x) = f(x + 2\pi)$ finner da
\begin{equation}
  \label{eq:2}
  \begin{split}
    a_{n} &= \frac{1}{\pi} \int_{0}^{2\pi} x^{2} \cos\nbrack{nx} dx \\
          &= \frac{1}{\pi} \bbrack{ x^{2} \frac{\sin nx}{n} }_{0}^{2\pi} - \frac{2}{n\pi} \int_{0}^{2\pi} x \sin nx dx \\
          &= \frac{2}{n^{2}\pi} x\cos nx \Big| _{0}^{2\pi} - \frac{2}{n^{2}\pi} \int_{0}^{2\pi} \cos nx dx \\
          &= \frac{2}{n^{2}} \cos 2n\pi + \frac{2}{n^{2}} - \frac{2}{n^{3}\pi} \nbrack{ \sin 2n\pi  - \sin 0 } \\
    &= \frac{4}{n^{2}}
  \end{split}
\end{equation}
finner også
\begin{equation}
  \label{eq:3}
  a_{0} = \frac{1}{2\pi} \int_{0}^{2\pi} x^{2} dx = \frac{1}{2\pi} \cdot \frac{8\pi^{3}}{3} = \frac{4\pi^{2}}{3}
\end{equation}
og til slutt
\begin{equation}
  \label{eq:4}
  \begin{split}
    b_{n} &= \frac{1}{\pi} \int_{0}^{2\pi} x^{2} \sin nx dx \\
          &= \frac{1}{\pi} \bbrack{ -x^{2} \frac{\cos nx}{n} \Big| _{0}^{2\pi} + \frac{2}{n} \int_{0}^{2\pi} x \cos nx dx } \\
          &= \frac{1}{\pi} \bbrack{ -\frac{4\pi^{2}}{n} + \frac{2}{n^{2}} x\sin nx \Big|_{0}^{2\pi} - \frac{2}{n^{2}} \int_{0}^{2\pi} \sin nx dx  } \\
          &= \frac{1}{\pi} \bbrack{ -\frac{4\pi^{2}}{n} - \frac{2}{n^{3}} \nbrack{ \cos 2n\pi - 1 } } \\
    &= -\frac{4\pi}{n}
  \end{split}
\end{equation}
og har dermed Fourierrekka
\begin{equation}
  \label{eq:5}
  f(x) \sim \frac{4\pi^{2}}{3} + \sum_{n=1}^{\infty} \nbrack{ \frac{4}{n^{2}} \cos nx - \frac{4\pi}{n} \sin nx }
\end{equation}


\section*{11.1.17}
Finner de samme uttrykkene
\begin{equation}
  \label{eq:6}
  \begin{split}
    a_{n} &= \frac{1}{\pi} \int_{-\pi}^{\pi} f(x) \cos nx dx \\
          &= \frac{2}{\pi} \int_{0}^{\pi} \bbrack{ -x\cos nx + \pi\cos nx}dx \\
          &= \frac{2}{\pi} \bbrack{ -\frac{x}{n} \sin nx \Big|_{0}^{\pi} + \frac{1}{n} \int_{0}^{\pi} \sin nx dx } \\
          &= \frac{2}{n^{2} \pi} \nbrack{ - \cos \pi n + 1 } \\
    &= \left\lbrace
      \begin{array}{ll}
        \frac{4}{n^{2}\pi} & , n = 2k-1 \\
        0 & , n = 2k
      \end{array} \right. , k\in \M{N}{}
  \end{split}
\end{equation}
ser så på
\begin{equation}
  \label{eq:7}
  a_{0} = \frac{1}{2\pi} \int_{-\pi}^{\pi} f(x) dx = \frac{\pi}{2}
\end{equation}
og til slutt
\begin{equation}
  \label{eq:8}
  b_{n} = \frac{1}{\pi} \int_{-\pi}^{\pi} f(x) \sin nx dx = 0
\end{equation}
siden den resulterende funksjonen blir odde, og har dermed
\begin{equation}
  \label{eq:9}
  f(x) \sim \frac{\pi}{2} + \sum_{n=1}^{\infty} \frac{4}{(2n-1)^{2}\pi} \cos \bbrack{ \nbrack{2n-1}x }
\end{equation}


\section*{11.1.21}
Ser at funksjonen er antisymmetrisk og har dermed $a_{n} = 0$ og finner dermed kun
\begin{equation}
  \label{eq:10}
  \begin{split}
    b_{n} &= \frac{2}{\pi} \int_{0}^{\pi} \bbrack{ -x \sin nx + \pi \sin nx}dx \\
    &= \frac{2}{\pi} \bbrack{ \frac{x}{n} \cos nx \Big|_{0}^{\pi} - \frac{1}{n} \int_{0}^{\pi} \cos nx dx - \frac{\pi}{n} \nbrack{ \cos n \pi - \cos 0 } } \\
    &= \frac{2}{n}
  \end{split}
\end{equation}
og har dermed
\begin{equation}
  \label{eq:11}
  f(x) \sim \sum_{n=1}^{\infty} \frac{2}{n} \sin nx
\end{equation}


\section*{11.2.1}
Vet at $e^{x}$ verken er odde eller like, mens $e^{-|x|}$ må være symmetrisk om $y=0$ og er dermed like, $x^{3}\cos nx$ er en odde funksjon ganget med en annen odde funksjon og er dermed like, $x^{2}\tan \pi x$ er en odde funksjon ganget med en like funksjone og er dermed odde, til slutt kan vi se på
\begin{equation}
  \label{eq:12}
  \sinh x - \cosh x = \frac{1}{2} \bbrack{ e^{x} - e^{-x} - e^{x} - e^{-x} } = -e^{-x}
\end{equation}
som verken er like eller odde.


\section*{11.2.10}
Ser at funksjonen er odde og trenger derfor kun å finne
\begin{equation}
  \label{eq:14}
  \begin{split}
    b_{n} &= \frac{2}{4} \int_{0}^{4} \bbrack{ -x \sin \nbrack{ \frac{n\pi x}{4} } + 4\sin \nbrack{ \frac{n\pi x}{4} } } dx \\
          &= \frac{2}{n\pi} \bbrack{ x\cos \nbrack{ \frac{n\pi x}{4} } \Big|_{0}^{4} - \int_{0}^{4} \cos \nbrack{ \frac{n\pi x}{4} }dx - 4 \nbrack{ \cos n\pi - 1 } } \\
    &= \frac{8}{n\pi}
  \end{split}
\end{equation}
og har dermed
\begin{equation}
  \label{eq:15}
  f(x) \sim \sum_{n=1}^{\infty} \frac{8}{n\pi} \sin \nbrack{ \frac{n\pi x}{4} }
\end{equation}


\section*{11.2.17}
Ser at funksjonen er odde og finner derfor ikke $b_{n}$
\begin{equation}
  \label{eq:16}
  \begin{split}
    a_{n} &= 2 \int_{0}^{1} \bbrack{ -x \cos n\pi x + \cos n\pi x } dx \\
          &= 2\bbrack{ -\frac{x}{n\pi} \sin n\pi x \Big|_{0}^{1} + \frac{1}{n\pi} \int_{0}^{1} \sin n\pi x dx } \\
          &= \frac{2}{n^{2}\pi^{2}} \nbrack{ -\cos n\pi + 1 } \\
    &= \left\lbrace
      \begin{array}{ll}
        \frac{2}{n^{2}\pi^{2}} & , n = 2k-1 \\
        0 & , n = 2k
      \end{array} \right. , k\in \M{N}{}
  \end{split}
\end{equation}
og har
\begin{equation}
  \label{eq:17}
  a_{0} = \frac{1}{2} \int_{-1}^{1} f(x) dx = \frac{1}{2}
\end{equation}
og har dermed
\begin{equation}
  \label{eq:18}
  f(x) \sim \frac{1}{2} + \sum_{n=1}^{\infty} \frac{2}{\pi^{2} \nbrack{2n-1}^{2}} \cos \bbrack{ \nbrack{2n-1} \pi x }
\end{equation}


\section*{11.2.24}
Finner først Fourier-cosinus-rekka til $f(x)$
\begin{equation}
  \label{eq:19}
  \begin{split}
    a_{n} &= \frac{2}{4} \int_{0}^{4} f(x) \cos \nbrack{ \frac{n\pi x}{4} }dx \\
    &= \frac{1}{2} \int_{2}^{4} \cos \nbrack{ \frac{n\pi x}{4} }dx \\
    &= \frac{2}{n\pi} \nbrack{ \sin n\pi - \sin \nbrack{ \frac{n\pi}{2} } } \\
    &= \left\lbrace
      \begin{array}{ll}
        -\frac{2}{n\pi} & , n = 4k - 3 \\
        \frac{2}{n\pi} & , n = 4k - 1 \\
        0 & , n = 2k \\
      \end{array} \right. , k\in \M{N}{}
  \end{split}
\end{equation}
og har $a_{0} = 2\cdot 2\cdot 1 / 8 = 1/2$
og har da Fourier-cosinus-rekka
\begin{equation}
  \label{eq:20}
  f(x) \sim \frac{1}{2} + \sum_{n=1}^{\infty} \cbrack{ \frac{2}{\nbrack{4n-3} \pi} \cos \bbrack{ \frac{\nbrack{4n-1} \pi x}{4} }
  -\frac{2}{\nbrack{4n-3}\pi} \cos \bbrack{ \frac{\nbrack{4n-3} \pi x}{4} } }
\end{equation}
finner så Fourier-sinus-rekka og har da
\begin{equation}
  \label{eq:21}
  b_{n} = \frac{1}{2} \int_{2}^{4} \sin \nbrack{\frac{n\pi x}{4}}dx = \frac{2}{n\pi} \bbrack{ -\cos n\pi + \cos \nbrack{ \frac{n\pi}{2} }}
  = \left\lbrace
    \begin{array}{ll}
      \frac{2}{n\pi} & , n = 2k-1 \\
      -\frac{4}{n\pi} & , n = 4k - 2 \\
      0 & , n = 4k
    \end{array} \right. , k \in \M{N}{}
\end{equation}


\section*{11.2.29}
Ser på $f(x) = \sin x$ for $0 < x < \pi$ og ser først på Fourier-sinus-rekka til $f(x)$ og ved å periodisk utvide til en odde funksjon får vi $\sin x, \forall x$ og har dermed at Fourier-sinus-rekka er gitt ved
\begin{equation}
  \label{eq:13}
  f(x) \sim \sin x
\end{equation}
ser så på Fourier-cosinus-rekka og har dermed
\begin{equation}
  \label{eq:22}
  \begin{split}
    a_{n} &= \frac{2}{\pi} \int_{0}^{\pi} \sin x \cos nx dx \\
          &= \frac{1}{2\pi} \int_{0}^{\pi} \bbrack{ ie^{-ix} - ie^{ix} } \bbrack{ e^{inx} + e^{-inx} }dx \\
          &= \frac{1}{2\pi} \int_{0}^{\pi} \bbrack{ ie^{ix(n-1)} + ie^{ix(-n-1)} - ie^{ix(n+1)} - ie^{ix(-n+1)} }dx \\
          &= \frac{1}{2\pi} \bbrack{ \frac{(-1)^{n-1} - 1}{n-1} + \frac{(-1)^{-n-1} - 1}{-n-1} - \frac{(-1)^{n+1} - 1}{n+1} - \frac{(-1)^{-n+1} - 1}{-n+1} } \\
          &= \frac{1}{2\pi} \bbrack{ \frac{2(-1)^{n-1} - 2}{n-1} - \frac{2(-1)^{n+1} - 2}{n+1} } \\
          &= \frac{1}{2\pi} \bbrack{ \frac{ 4(-1)^{n+1} - 4 }{n^{2} - 1} } \\
          &= \frac{2\bbrack{ (-1)^{n+1} - 1 }}{\pi \nbrack{ n^{2} - 1 }}
  \end{split}
\end{equation}

ser så at denne rekka ikke er definert for $n=1$ og finner da
\begin{equation}
  \label{eq:23}
  a_{1} = \frac{2}{\pi} \int_{0}^{\pi} \sin x \cos x dx = \frac{1}{\pi} \int_{0}^{\pi} \sin 2x dx = 0
\end{equation}
og finner så
\begin{equation}
  \label{eq:24}
  a_{0} = \frac{1}{\pi} \int_{0}^{\pi} \sin x dx = \frac{1}{\pi} \bbrack{ -\cos \pi + 1 } = \frac{2}{\pi}
\end{equation}
og har dermed Fourier-cosinus-rekka
\begin{equation}
  \label{eq:25}
  f(x) \sim \frac{2}{\pi} + \sum_{n=2}^{\infty} \frac{2\bbrack{ (-1)^{n+1} - 1 }}{\pi \nbrack{ n^{2} - 1 }} \cos nx
\end{equation}



\section*{11.3.15}




\end{document}
