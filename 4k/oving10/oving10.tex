\documentclass{report}

\usepackage{amsfonts, amsmath, amssymb, amsthm}
\usepackage[margin=1.0in]{geometry}
\usepackage{hyperref}
\usepackage{float}
\usepackage{fancyhdr}
\usepackage{graphicx}
\usepackage{mathrsfs}
\usepackage{comment}

\newcommand{\M}[2]{\mathbb{#1}^{#2}}
\newcommand{\twovector}[2]{\left[ \begin{array}{c} #1 \\ #2 \\ \end{array} \right]}
\newcommand{\threevector}[3]{\left[ \begin{array}{c} #1 \\ #2 \\ #3 \\ \end{array} \right]}
\newcommand{\fourvector}[4]{\left[ \begin{array}{c} #1 \\ #2 \\ #3 \\ #4 \\ \end{array} \right]}
\newcommand{\fivevector}[5]{\left[ \begin{array}{c} #1 \\ #2 \\ #3 \\ #4 \\ #5 \\ \end{array} \right]}
\newcommand{\sixvector}[6]{\left[ \begin{array}{c} #1 \\ #2 \\ #3 \\ #4 \\ #5 \\ #6 \\ \end{array} \right]}
\newcommand{\sevenvector}[7]{\left[ \begin{array}{c} #1 \\ #2 \\ #3 \\ #4 \\ #5 \\ #6 \\ #7 \\ \end{array} \right]}
\newcommand{\eightvector}[8]{\left[ \begin{array}{c} #1 \\ #2 \\ #3 \\ #4 \\ #5 \\ #6 \\ #7 \\ #8 \\ \end{array} \right]}
\newcommand{\ninevector}[9]{\left[ \begin{array}{c} #1 \\ #2 \\ #3 \\ #4 \\ #5 \\ #6 \\ #7 \\ #8 \\ #9 \\ \end{array} \right]}
\newcommand{\nbrack}[1]{\left( #1 \right)}
\newcommand{\bbrack}[1]{\left[ #1 \right]}
\newcommand{\cbrack}[1]{\left\lbrace #1 \right\rbrace}
\newcommand{\abrack}[1]{\left< #1 \right>}
\newcommand{\linebrack}[1]{\left| #1 \right|}
\newcommand{\twomatrix}[4]{\bbrack{
    \begin{array}{cc}
      #1 & #2 \\
      #3 & #4 \\
    \end{array}
  }
}
\newcommand{\threematrix}[9]{\bbrack{
    \begin{array}{ccc}
      #1 & #2 & #3 \\
      #4 & #5 & #6 \\
      #7 & #8 & #9 \\
    \end{array}
  }
}
\newcommand{\Lplc}[1]{\mathscr{L}\bbrack{ #1 } (s)}
\newcommand{\iLplc}[1]{\mathscr{L}^{-1}\bbrack{ #1 } (t)}
\newcommand{\im}{\text{Im}}
\newcommand{\re}{\text{Re}}

\title{Innlevering 10}
\author{Jacob Oliver Bruun}
\date{\today}

\makeatletter
\let\inserttitle\@title
\let\insertauthor\@author
\makeatother

\pagestyle{fancy}
\chead{\insertauthor}
\lhead{\inserttitle}
\rhead{\today}

\parindent 0ex

\begin{document}

\section*{14.3.2}
Ser på
\begin{equation}
  \label{eq:1}
  \oint_{\mathcal{C}} \frac{z^{2}}{z^{2} - 1} dz, \;\;\; \mathcal{C} : |z - 1 - i| = \frac{\pi}{2}
\end{equation}
kan ved delbrøksoppspaltning finne brøken forenklet og dermed ser på polynomdivisjonen
\begin{equation}
  \label{eq:3}
  \nbrack{z^{2}} / \nbrack{z^{2} - 1} = 1 + \frac{1}{z^{2} - 1}
\end{equation}
og kan så videre se
\begin{equation}
  \label{eq:2}
  \begin{split}
    \frac{1}{z^{2} - 1} &= \frac{A}{z-1} + \frac{B}{z+1} \\
    \frac{1}{z^{2} - 1} &= \frac{A(z+1) + B(z-1)}{z^{2} - 1} \\
    1 &= Az + A + Bz - B \Rightarrow A = \frac{1}{2}, B = -\frac{1}{2}
  \end{split}
\end{equation}
og har dermed
\begin{equation}
  \label{eq:4}
  \begin{split}
    \oint_{\mathcal{C}} \frac{z^{2}}{z^{2} - 1} dz
    &= \oint_{\mathcal{C}} \bbrack{ 1 + \frac{1}{2(z-1)} - \frac{1}{2(z+1)} }dz \\
    &= \oint_{\mathcal{C}} dz + \frac{1}{2} \oint_{\mathcal{C}} \frac{1}{z-1} dz - \frac{1}{2} \oint_{\mathcal{C}} \frac{1}{z+1} dz \\
    &= 0 + \frac{1}{2} \cdot 2\pi i - 0 \\
    &= \pi i
  \end{split}
\end{equation}
siden $z=1$ ligger i sirkelen, mens $z=-1$ ikke ligger innenfor sirkelen.


\section*{14.3.13}
Ser på integralet
\begin{equation}
  \label{eq:5}
  \oint_{\mathcal{C}} \frac{z+2}{z-2} dz, \;\;\; \mathcal{C} : |z - 1| = 2
\end{equation}
ser at $z=2$ ligger innenfor sirkelen og har dermed
\begin{equation}
  \label{eq:6}
  \oint_{\mathcal{C}} \frac{z+2}{z-2} dz = 2\pi i (2 + 2) = 8\pi i
\end{equation}


\section*{14.3.18}
Ser på integralet
\begin{equation}
  \label{eq:7}
  \oint_{\mathcal{C}} \frac{\sin z}{4z^{2} - 8iz} dz = \frac{1}{4} \oint_{\mathcal{C}} \frac{\sin z}{z^{2} - 2iz} dz
\end{equation}
hvor $\mathcal{C}$ er kvadratet med hjørnene $\pm 3i, \pm 3$ mot klokka og $\pm i, \pm 1$ med klokka, og ser så
\begin{equation}
  \label{eq:8}
  \begin{split}
    \frac{1}{z^{2} - 2iz} &= \frac{A}{z} + \frac{B}{z - 2i} \\
    \frac{1}{z^{2} - 2iz} &= \frac{Az - 2Ai + Bz}{z^{2} - 2iz} \\
    1 &= Az - 2Ai + Bz \Rightarrow A = \frac{i}{2}, B = -\frac{i}{2}
  \end{split}
\end{equation}
og dermed
\begin{equation}
  \label{eq:9}
  \oint_{\mathcal{C}} \frac{\sin z}{4z^{2} - 8iz} dz = \frac{i}{8} \oint_{\mathcal{C}} \bbrack{ \frac{\sin z}{z} - \frac{\sin z}{z - 2i} }dz
  = -\frac{i}{4} \pi i \sin 2i = \frac{\pi i}{8} \bbrack{ e^{2} - e^{-2} }
\end{equation}


\section*{14.4.2}
Ser på intagralet
\begin{equation}
  \label{eq:10}
  \oint_{\mathcal{C}} \frac{z^{6}}{(2z-1)^{6}} dz
\end{equation}
hvor $\mathcal{C}$ er enhetssirkelen og med $f(z) = z^{6}$ er
\begin{equation}
  \label{eq:11}
  f^{(5)}(z) = 6! z
\end{equation}
og har da
\begin{equation}
  \label{eq:12}
  \oint_{\mathcal{C}} \frac{z^{6}}{(2z-1)^{6}} dz = \oint_{\mathcal{C}} \frac{z^{6}}{2^{1/6}(z-1/2)^{6}} = \frac{2\pi i}{2^{1/6}\cdot 5!} 6! \cdot \frac{1}{2} = 2^{5/6} \cdot 3\pi i
\end{equation}


\section*{14.4.7}
\section*{14.4.16}
\section*{15.1.17}
\section*{15.1.18}
\section*{15.2.5}
\section*{15.2.10}
\section*{15.2.14}



\end{document}
