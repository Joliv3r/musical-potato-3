\documentclass{report}

\usepackage{amsfonts, amsmath, amssymb, amsthm}
\usepackage[margin=1.0in]{geometry}
\usepackage{hyperref}
\usepackage{float}
\usepackage{fancyhdr}
\usepackage{graphicx}
\usepackage{mathrsfs}
\usepackage{comment}
\usepackage{mathtools}

\newcommand{\M}[2]{\mathbb{#1}^{#2}}
\newcommand{\twovector}[2]{\left[ \begin{array}{c} #1 \\ #2 \\ \end{array} \right]}
\newcommand{\threevector}[3]{\left[ \begin{array}{c} #1 \\ #2 \\ #3 \\ \end{array} \right]}
\newcommand{\fourvector}[4]{\left[ \begin{array}{c} #1 \\ #2 \\ #3 \\ #4 \\ \end{array} \right]}
\newcommand{\fivevector}[5]{\left[ \begin{array}{c} #1 \\ #2 \\ #3 \\ #4 \\ #5 \\ \end{array} \right]}
\newcommand{\sixvector}[6]{\left[ \begin{array}{c} #1 \\ #2 \\ #3 \\ #4 \\ #5 \\ #6 \\ \end{array} \right]}
\newcommand{\sevenvector}[7]{\left[ \begin{array}{c} #1 \\ #2 \\ #3 \\ #4 \\ #5 \\ #6 \\ #7 \\ \end{array} \right]}
\newcommand{\eightvector}[8]{\left[ \begin{array}{c} #1 \\ #2 \\ #3 \\ #4 \\ #5 \\ #6 \\ #7 \\ #8 \\ \end{array} \right]}
\newcommand{\ninevector}[9]{\left[ \begin{array}{c} #1 \\ #2 \\ #3 \\ #4 \\ #5 \\ #6 \\ #7 \\ #8 \\ #9 \\ \end{array} \right]}
\newcommand{\nbrack}[1]{\left( #1 \right)}
\newcommand{\bbrack}[1]{\left[ #1 \right]}
\newcommand{\cbrack}[1]{\left\lbrace #1 \right\rbrace}
\newcommand{\abrack}[1]{\left< #1 \right>}
\newcommand{\linebrack}[1]{\left| #1 \right|}
\newcommand{\twomatrix}[4]{\bbrack{
    \begin{array}{cc}
      #1 & #2 \\
      #3 & #4 \\
    \end{array}
  }
}
\newcommand{\threematrix}[9]{\bbrack{
    \begin{array}{ccc}
      #1 & #2 & #3 \\
      #4 & #5 & #6 \\
      #7 & #8 & #9 \\
    \end{array}
  }
}
\newcommand{\Lplc}[1]{\mathscr{L}\bbrack{ #1 } (s)}
\newcommand{\iLplc}[1]{\mathscr{L}^{-1}\bbrack{ #1 } (t)}
\newcommand{\im}{\text{Im}}
\newcommand{\re}{\text{Re}}
\newcommand{\limtoinf}[1]{\lim_{#1 \rightarrow \infty}}

\title{Innlevering 10}
\author{Jacob Oliver Bruun}
\date{\today}

\makeatletter
\let\inserttitle\@title
\let\insertauthor\@author
\makeatother

\pagestyle{fancy}
\chead{\insertauthor}
\lhead{\inserttitle}
\rhead{\today}

\parindent 0ex

\begin{document}

\section*{14.3.2}
Ser på
\begin{equation}
  \label{eq:1}
  \oint_{\mathcal{C}} \frac{z^{2}}{z^{2} - 1} dz, \;\;\; \mathcal{C} : |z - 1 - i| = \frac{\pi}{2}
\end{equation}
kan ved delbrøksoppspaltning finne brøken forenklet og dermed ser på polynomdivisjonen
\begin{equation}
  \label{eq:3}
  \nbrack{z^{2}} / \nbrack{z^{2} - 1} = 1 + \frac{1}{z^{2} - 1}
\end{equation}
og kan så videre se
\begin{equation}
  \label{eq:2}
  \begin{split}
    \frac{1}{z^{2} - 1} &= \frac{A}{z-1} + \frac{B}{z+1} \\
    \frac{1}{z^{2} - 1} &= \frac{A(z+1) + B(z-1)}{z^{2} - 1} \\
    1 &= Az + A + Bz - B \Rightarrow A = \frac{1}{2}, B = -\frac{1}{2}
  \end{split}
\end{equation}
og har dermed
\begin{equation}
  \label{eq:4}
  \begin{split}
    \oint_{\mathcal{C}} \frac{z^{2}}{z^{2} - 1} dz
    &= \oint_{\mathcal{C}} \bbrack{ 1 + \frac{1}{2(z-1)} - \frac{1}{2(z+1)} }dz \\
    &= \oint_{\mathcal{C}} dz + \frac{1}{2} \oint_{\mathcal{C}} \frac{1}{z-1} dz - \frac{1}{2} \oint_{\mathcal{C}} \frac{1}{z+1} dz \\
    &= 0 + \frac{1}{2} \cdot 2\pi i - 0 \\
    &= \pi i
  \end{split}
\end{equation}
siden $z=1$ ligger i sirkelen, mens $z=-1$ ikke ligger innenfor sirkelen.


\section*{14.3.13}
Ser på integralet
\begin{equation}
  \label{eq:5}
  \oint_{\mathcal{C}} \frac{z+2}{z-2} dz, \;\;\; \mathcal{C} : |z - 1| = 2
\end{equation}
ser at $z=2$ ligger innenfor sirkelen og har dermed
\begin{equation}
  \label{eq:6}
  \oint_{\mathcal{C}} \frac{z+2}{z-2} dz = 2\pi i (2 + 2) = 8\pi i
\end{equation}


\section*{14.3.18}
Ser på integralet
\begin{equation}
  \label{eq:7}
  \oint_{\mathcal{C}} \frac{\sin z}{4z^{2} - 8iz} dz = \frac{1}{4} \oint_{\mathcal{C}} \frac{\sin z}{z^{2} - 2iz} dz
\end{equation}
hvor $\mathcal{C}$ er kvadratet med hjørnene $\pm 3i, \pm 3$ mot klokka og $\pm i, \pm 1$ med klokka, og ser så
\begin{equation}
  \label{eq:8}
  \begin{split}
    \frac{1}{z^{2} - 2iz} &= \frac{A}{z} + \frac{B}{z - 2i} \\
    \frac{1}{z^{2} - 2iz} &= \frac{Az - 2Ai + Bz}{z^{2} - 2iz} \\
    1 &= Az - 2Ai + Bz \Rightarrow A = \frac{i}{2}, B = -\frac{i}{2}
  \end{split}
\end{equation}
og dermed
\begin{equation}
  \label{eq:9}
  \oint_{\mathcal{C}} \frac{\sin z}{4z^{2} - 8iz} dz = \frac{i}{8} \oint_{\mathcal{C}} \bbrack{ \frac{\sin z}{z} - \frac{\sin z}{z - 2i} }dz
  = -\frac{i}{4} \pi i \sin 2i = \frac{\pi i}{8} \bbrack{ e^{2} - e^{-2} }
\end{equation}


\section*{14.4.2}
Ser på intagralet
\begin{equation}
  \label{eq:10}
  \oint_{\mathcal{C}} \frac{z^{6}}{(2z-1)^{6}} dz
\end{equation}
hvor $\mathcal{C}$ er enhetssirkelen og med $f(z) = z^{6}$ er
\begin{equation}
  \label{eq:11}
  f^{(5)}(z) = 6! z
\end{equation}
og har da
\begin{equation}
  \label{eq:12}
  \oint_{\mathcal{C}} \frac{z^{6}}{(2z-1)^{6}} dz = \oint_{\mathcal{C}} \frac{z^{6}}{2^{1/6}(z-1/2)^{6}} = \frac{2\pi i}{2^{1/6}\cdot 5!} 6! \cdot \frac{1}{2} = 2^{5/6} \cdot 3\pi i
\end{equation}


\section*{14.4.7}
Ser på integralet
\begin{equation}
  \label{eq:13}
  \oint_{\mathcal{C}} \frac{\cos z}{z^{2n+1}} dz, \;\;\; n = 0, 1, \dots
\end{equation}
med $\mathcal{C}$ enhetssirkelen, kan da se at
\begin{equation}
  \label{eq:14}
  f^{(2n)} = (-1)^{n} \cos z
\end{equation}
og har dermed
\begin{equation}
  \label{eq:15}
  \oint_{\mathcal{C}} \frac{\cos z}{z^{2n+1}} dz = (-1)^{n} \cdot \frac{2\pi i}{(2n)!}
\end{equation}


\section*{14.4.16}
Ser på integralet
\begin{equation}
  \label{eq:16}
  \oint_{\mathcal{C}} \frac{e^{4z}}{z(z-2i)^{2}} dz
\end{equation}
med $\mathcal{C}$ er gitt ved $|z-i| = 3$ mot klokka og $|z| = 1$ med klokka og kan skrive
\begin{equation}
  \label{eq:17}
  \begin{split}
    \frac{1}{z(z-2i)^{2}} &= \frac{A}{z} + \frac{B}{z-2i} + \frac{C}{(z-2i)^{2}} \\
    1 &= A(z-2i)^{2} + Bz(z-2i) + Cz \\
    1 &= Az^{2} - 4Azi - 4A + Bz^{2} - 2Bzi + Cz
  \end{split}
\end{equation}
og har fra dette $A = -1/4, B = 1/4, C = -i/2$ og dermed
\begin{equation}
  \label{eq:18}
  \begin{split}
    \oint_{\mathcal{C}} \frac{e^{4z}}{z(z-2i)^{2}}dz
    &= \oint_{\mathcal{C}} \bbrack{ \frac{e^{4z}}{4(z-2i)} - \frac{e^{4z}}{4z} - \frac{ie^{4z}}{2(z-2i)^{2}}} dz \\
    &= \frac{1}{4} \oint_{\mathcal{C}} \frac{e^{4z}}{z-2i} dz - \frac{1}{4} \oint_{\mathcal{C}} \frac{e^{4z}}{z} dz - \frac{i}{2} \oint_{\mathcal{C}} \frac{e^{4z}}{(z-2i)^{2}} \\
    &= \frac{1}{2} \pi i e^{8i} - 0 + 4\pi e^{8i} \\
    &= \nbrack{ 4 + \frac{1}{2}i }\pi e^{8i}
  \end{split}
\end{equation}


\section*{15.1.17}
Ser på rekka
\begin{equation}
  \label{eq:19}
  \sum_{n=2}^{\infty} \frac{(-i)^{n}}{\ln n}
\end{equation}
kan skrive om til
\begin{equation}
  \label{eq:20}
  f(z) = \sum_{n=2}^{\infty} \frac{z^{n}}{\ln n}
\end{equation}
og kan finne konvergensradius ved Cauchy-Hardamard


\section*{15.1.18}
Ser på rekka
\begin{equation}
  \label{eq:21}
  \sum_{n=1}^{\infty} n^{2} \nbrack{ \frac{i}{4} }^{n}
\end{equation}
og ser ved å bruke rottesten
\begin{equation}
  \label{eq:22}
  \lim_{n \rightarrow \infty} \sqrt[n]{|z_{n}|} = \lim_{n \rightarrow \infty} n^{\frac{2}{n}} \frac{1}{4} = \frac{1}{4} \lim_{n \rightarrow \infty} n^{\frac{2}{n}} = \frac{1}{4} \lim_{n \rightarrow \infty} \exp \cbrack{\ln n^{\frac{2}{n}}} = \frac{1}{4} \exp \cbrack{ \lim_{n \rightarrow \infty} \frac{2 \ln n}{n} } = \frac{1}{4} < 1
\end{equation}
og dermed konvergerer rekka.


\section*{15.2.5}
Ser på en rekke
\begin{equation}
  \label{eq:23}
  \sum_{n} a_{n} z^{n}
\end{equation}
med konvergensradius $R$ og ser så på
\begin{equation}
  \label{eq:24}
  \sum_{n} a_{n} z^{2n} = \sum
\end{equation}
og ved å se på forholdstesten finner vi
\begin{equation}
  \label{eq:26}
  \limtoinf{n} \left| \frac{a_{n+1}}{a_{n}}z^{2} \right| = L z^{2}
\end{equation}
ved å anta $L\neq 0$ har vi $L > 0$ og ved forholdstesten har vi konvergens ved
\begin{equation}
  \label{eq:27}
  Lz^{2} < 1 \Rightarrow z < \frac{1}{\sqrt{L}}
\end{equation}
ved Cauchy-Hardamard vet vi at
\begin{equation}
  \label{eq:28}
  R = \limtoinf{n} \left| \frac{a_{n}}{a_{n+1}} \right| = \frac{1}{L}
\end{equation}
og dermed konvergens for
\begin{equation}
  \label{eq:29}
  z < \sqrt{R}
\end{equation}


\section*{15.2.10}
Ser rekka
\begin{equation}
  \label{eq:30}
  \sum_{n=0}^{\infty} \frac{(z - 2i)^{n}}{n^{n}}
\end{equation}
har sentrum i $z = 2i$ og kan bruke Cauchy-Hardamard
\begin{equation}
  \label{eq:31}
  \left| \frac{(n+1)^{n+1}}{n^{n}} \right| \xrightarrow{n \to \infty} \infty
\end{equation}
og har dermed uendelig konvergensradius og rekka konvergerer for alle $z \in \M{C}{}$.


\section*{15.2.14}
Ser på rekka
\begin{equation}
  \label{eq:32}
  \sum_{n=0}^{\infty} \frac{(-1)^{n}}{4^{2n}(n!)^{2}} z^{2n}
\end{equation}
ser at sentrum er i $z = 0$ og fra beviset tidligere har vi konvergens i $\sqrt{R}$ dersom $R$ er gitt av Cauchy-Hardamard
\begin{equation}
  \label{eq:33}
  R = \limtoinf{n} \left| \frac{(-1)^{n} 4^{2(n+1)} \bbrack{ (n+1)! }^{2}}{4^{2n}(n!)^{2} (-1)^{n+1}} \right|
  = \limtoinf{n} 4^{2} \nbrack{ n+1 }^{2} \to \infty
\end{equation}
fra Cauchy-Hardamard vil dette gjelde for uendelig store konvergensradiuser for $z^{2n}$ også og vi har dermed en uendelig stor konvergensradius.


\end{document}
